\documentclass{article}

\usepackage{physics} % Handy shortcuts like \pdv, \dd and much more
\usepackage{geometry} % smaller margins, can be adjusted if given arguments
\usepackage{siunitx} % the \si environment for units
\usepackage{mathtools} % The dcases environment, prettier than just cases
\usepackage{tikz} % For drawing picures
\usepackage{wrapfig} % Wrapping text around figures


\title{Exercise 2 - TFY4345 Classical Mechanics}
\date{2020}

\begin{document}
    \maketitle
    \section{Damped oscillator}
        A particle with mass $m$ moves with a low velocity $\dot x$. The particle is a damped oscillator with center in the origin and spring constant $k$. The frictional force is 
        \begin{equation*}
            F_f = - \pdv{\mathcal{F}}{\dot x},
        \end{equation*}
        where $\mathcal{F}$ is Rayleigh's dissipation function. \\ \\
        (a) Show that $\mathcal{F} \propto \dot x^2$, and that the viscous energy loss per unit time $\dot W_f$ can be written as $\dot W_f = 2 \mathcal{F}$. \\ \\
        (b) Assume then that $\mathcal{F} = 3 \pi \mu a \dot x^2$, where $\mu$ is the dynamic viscosity and $a$ the particle radius. Start from Lagrange's equation, and show that the equation for motion can be written as 
        \begin{equation*}
            \ddot x + 2 \lambda \dot x + \omega^2_0 x = 0.
        \end{equation*}
        Express $\lambda$ and $\omega_0$ in terms of the constants $k,\, m,\, \mu $ and $a$.\\ \\
        (c) Solve the equation for $x(t)$ when $\lambda / \omega_0 \ll 1$, assuming $x(0) = x_0$ and $\dot x(0) = 0$, and show that one approximately has 
        \begin{equation*}
            \overline{\dot W_f} = m \lambda \left(\omega_0 x_0\right)^2 e^{-2 \lambda t},
        \end{equation*}
        where the bar denotes time average.
        

    \section{Operator identities}
        Use the Levi-Civita tensor to prove the following vector-operator relation:
        \begin{equation*}
            \nabla \times \left( \nabla \times \mathbf{A} \right) = \nabla \left(\nabla \cdot \mathbf{A}\right) - \nabla^2 \mathbf{A}.
        \end{equation*}
        (Note: There will be no such assignment in the exam where one has to play with this tensor.)

    \section{Shortest path in polar coordinates}
        Show using polar coordinates and the Euler equation that the shortest distance between two points is a straight line, $r = b / \sin(\phi), \, b = \mathrm{const}$. In this case, the Euler equation is
        \begin{equation*}
            \pdv{f}{y} - \dv{x} \pdv{f}{y'} = 0.
        \end{equation*}
        (Hint: here, you have to choose which of the variables $r, \phi$ are the parameter, $x$ in the above equation or time in physical problems, and which is the free variable, as $y$ is in the equation above. Choose $r$ to be the parameter.)

    \section{Forces of constraint}
        Consider a mathematical pendulum in two dimensions. 
        Evaluate the equations of motion, and find the tension force within the pendulum string by using the Euler equations and an undetermined multiplier $\lambda$. 
        Interpret the resulting force of constraint. 
        \\ \\ 
        (Hint: Set the constraint such that the wire length $\ell$ is constant.)


\end{document}

