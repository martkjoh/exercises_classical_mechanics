\documentclass{article}

\usepackage{physics} % Handy shortcuts like \pdv, \dd and much more
\usepackage{geometry} % smaller margins, can be adjusted if given arguments
\usepackage{siunitx} % the \si environment for units
\usepackage{mathtools} % The dcases environment, prettier than just cases
\usepackage{tikz} % For drawing picures
\usepackage{wrapfig} % Wrapping text around figures


\title{Exercise 2 solutions - TFY4345 Classical Mechanics}
\date{2020}

\newcommand{\Eff}{\mathcal{F}}
\newcommand{\Ce}{\mathcal{C}}
\begin{document}
    \maketitle
    \section{Damped oscillator}
        (a) The frictional force is 
        \begin{equation*}
            F_f = -\pdv{\mathcal{F}}{\dot x}.
        \end{equation*}
        The work done by friction is force times distance, so the work per unit time is
        \begin{equation*}
            \dot W_f = - F_f \dot x = \pdv{\mathcal{F}}{\dot x} \dot x \implies \mathcal{F} = C \dot x ^2.
        \end{equation*}
        (As $\mathcal{F}$ is a (velocity) potential, we can dismiss any constants, just as with regular potentials.) This means that
        \begin{equation*}
            \dot W_f = 2 C \dot x^2 = 2 \mathcal{F}. 
        \end{equation*}
        (b) The Lagrangian with a velocity-dependent potential is 
        \begin{equation*}
            \dv{t} \pdv{L}{\dot x} - \pdv{L}{x} + \pdv{\Eff}{\dot x} = 0.
        \end{equation*}
        Inserting the Lagrangian for a harmonic oscillator, 
        \begin{equation*}
            L = \frac{1}{2} m \dot x^2 - \frac{1}{2} k x^2,
        \end{equation*}
        and the given velocity potential $\Eff = 3 \pi \mu a \dot x^2$, we get
        \begin{align*}
            &\dv{t} \pdv{L}{\dot x} = m \ddot x, \, \pdv{L}{x} = -k x, \, \pdv{\Eff}{\dot x} = 6 \pi \mu a \dot x, \\
            &\implies m \ddot x + 6 \pi \mu a + k x = 0,
        \end{align*}
        or
        \begin{equation*}
            \ddot x + 2 \lambda \dot x + \omega^2_0x = 0, \quad \lambda = \frac{3 \pi \mu a}{m}, \, \omega_0 = \sqrt{\frac{k}{m}}.
        \end{equation*}
        (c) 
        If we assume the solution to be of the form
        \begin{equation*}
            x(t) = A e^{\omega_a t} + B e^{\omega_b t},
        \end{equation*}
        we get
        \begin{equation*}
            A(\omega_0^2 + 2 \lambda \omega_a + \omega^2)e^{\omega_a t} + B(\omega_0^2 + 2 \lambda \omega_b + \omega_b^2)e^{\omega_b t} = 0,
        \end{equation*}
        so
        \begin{equation*}
            \omega_{a/b} = -\lambda \pm \sqrt{\lambda^2 - \omega_0^2}.
        \end{equation*}
        This gives us
        \begin{equation*}
            x(t) = e^{-\lambda t} \left( A \exp \left[\omega_0 t \sqrt{(\lambda / \omega_0)^2 - 1} \right] + B \exp \left[- \omega_0 t \sqrt{(\lambda / \omega_0)^2 - 1} \right] \right).
        \end{equation*}
        Now, as $\lambda/\omega_0 \ll 1$, we get that $\sqrt{(\lambda / \omega_0) - 1} \approx i$. This means that, after applying the initial conditions, we get the solution
        \begin{equation*}
            x(t) = x_0 e^{-\lambda t} \cos(\omega_0 t).
        \end{equation*}
        The instantaneous energy dissipation is therefore 
        \begin{equation*}
            \dot W_f = F_f \dot x = 2 \lambda m \dot x^2 = 2 \lambda m (\omega_0x_0)^2 e^{-2\lambda t} \cos^2(\omega_0 t).
        \end{equation*}
        If we then take the average over a period $2 \pi / \omega_0$, we can assume the exponential is more or less constant (as $\lambda \ll \omega_0$), so the time averaged dissipation is 
        \begin{equation*}
            \overline{\dot W_f} = \frac{\omega_0}{2 \pi} \int_0^{2 \pi} \dd t \, 2 \lambda m (\omega_0x_0)^2 e^{-2\lambda t} \cos^2(\omega_0 t) \approx \frac{\lambda m (\omega_0x_0)^2}{\pi} e^{-2 \lambda t} \int_0^{2 \pi} \dd x \, \cos^2(x) =  m \lambda (\omega_0 x_0)^2 e^{-2\lambda t}.
        \end{equation*}

    \section{Operator identities}
        The Einstein summation convention is used throughout this exercise, so repeated indices are summed over. The $i$'th component of the curl of the curl of $\mathbf{A}$ can be written
        \begin{equation*}
            [\nabla \times (\nabla \times \mathbf{A})]_i = \varepsilon_{ijk} \partial_j[\nabla \times \mathbf{A}]_k = \varepsilon_{ijk} \partial_j \varepsilon_{klm} \partial_l A_m = \varepsilon_{ijk} \varepsilon_{klm} \partial_j \partial_l A_m.
        \end{equation*}
        Using the fact that $\varepsilon_{ijk} = \varepsilon_{kij}$, and the identity $\varepsilon_{ijk}\varepsilon_{imn} = \delta_{jm} \delta_{kn} - \delta_{jn}\delta_{km}$, this becomes
        \begin{equation*}
            [\nabla \times (\nabla \times \mathbf{A})]_i = (\delta_{il}\delta_{jm} - \delta_{im}\delta_{jl}) \partial_j \partial_l A_m = \partial_j \partial_i A_j - \partial_j \partial_j A_i = \nabla \cdot (\nabla_i \mathbf{A}) - (\nabla \cdot \nabla) A_i,
        \end{equation*}
        or
        \begin{equation*}
            \nabla \times (\nabla \times \mathbf{A}) = \nabla \cdot (\nabla \mathbf{A}) - \nabla^2 \mathbf{A}.
        \end{equation*}

    \section{Shortest path in polar coordinates}
        A distance element in polar coordinates are given by
        \begin{equation*}
            \dd s = \sqrt{\dd r^2 + r^2 \dd \varphi ^2} = \sqrt{1 + r^2 \varphi'^2} \dd r.
        \end{equation*}
        This means the distance along a curve $\Ce$ is
        \begin{equation*}
            s = \int_\Ce \dd s = \int_\Ce \sqrt{1 + r^2 \varphi'^2} \dd r.
        \end{equation*}
        With $r$ as our parameter, the Euler equation reads
        \begin{equation*}
            \dv{r} \pdv{f}{\varphi'} - \pdv{f}{\varphi} = 0,
        \end{equation*}
        where $f(\varphi', r) = \sqrt{1 + r^2 \varphi'^2}$ is what we are trying to minimize. We see that
        \begin{equation*}
            \pdv{f}{\varphi} = 0 \implies \dv{r} \pdv{f}{\varphi'} = 0,
        \end{equation*}
        so 
        \begin{align*}
            &\pdv{\varphi'} \sqrt{1 + r^2 \varphi'^2} = \frac{- r^2 \varphi'}{\sqrt{1 + r^2 \varphi'^2} } = a \\
            &\implies \frac{\varphi^2 r^2}{a} = \sqrt{1 + r^2 \varphi'^2} \\
            &\implies \frac{\varphi^4 r^4}{a^2} = 1 + r^2 \varphi'^2 \\
            &\implies \left( \frac{r^2}{a^2} - 1 \right)r^2 \varphi'^2 = 1.
        \end{align*}
        Defining $b^2 = 1 / a^2$, we get 
        \begin{align*}
            &\varphi'^2 = \frac{1}{r^2(b^2 r^2 - 1)} \\
            & \implies \varphi' = \pm \frac{1}{r \sqrt{b^2 r^2 - 1}} \\
            & \implies \varphi(r) = \pm \int_{r_0}^r \frac{\dd r}{r \sqrt{b^2 r^2 - 1}}.
        \end{align*}
        This integral can be found in a table, so we get
        \begin{equation*}
            \varphi(r) = \arcsin\left( \frac{-2}{r \sqrt{4b^2}}\right) + c.
        \end{equation*}
        Setting $c=0$, we get
        \begin{equation*}
            \sin(\varphi) = \pm \frac{1}{br} \implies r = \frac{a}{\sin(\varphi)}, \,r\geq 0.
        \end{equation*}

    \section{Forces of constraint}
        When using the method of undetermined multiplier, we do not assume $\dot r = 0$, but rather enforce this by a constraint. As we have found before, the kinetic energy is 
        \begin{equation*}
            T = \frac{1}{2}m \left(\dot r^2 + (r \dot \beta)^2\right),
        \end{equation*}
        and the potential energy is 
        \begin{equation*}
            V = -mgr \cos(\beta).
        \end{equation*}
        The constraint needed for a pendulum of length $\ell$ is $\ell - r = 0$, so the Lagrangian becomes
        \begin{equation*}
            L = \frac{1}{2}m \left(\dot r^2 + (r \dot \beta)^2\right) + mgr \cos(\beta) + \lambda \left( \ell - r\right).
        \end{equation*}
        This gives the equations of motion
        \begin{align}
            m \ddot r - m r \dot \beta^2 - m g \cos(\beta) + \lambda = 0, \\
            m r^2 \dot \beta + mg r \sin(\beta) = 0,
        \end{align}
        as well as the constraint $\ell - r = 0$. This implies $\dot r = \ddot r = 0$. The second equation gives, as before,
        \begin{equation*}
            \dot \beta + \frac{g}{\ell} \sin(\beta) = 0,
        \end{equation*}
        while the first gives 
        \begin{equation*}
            \lambda = m(\ell \dot \beta + g \cos(\beta)).
        \end{equation*}
        The tension in a pendulum string is the sum of the component of the gravitational force parallel to the string ($mg \cos(\beta)$), and the centripetal force acing on the mass to keep it going in a circle ($m \ell \dot \beta^2$). Thus, we recognize $\lambda$ as the tension in the pendulum string.

\end{document}
