\documentclass{article}

\usepackage{physics} % Handy shortcuts like \pdv, \dd and much more
\usepackage{geometry} % smaller margins, can be adjusted if given arguments
\usepackage{siunitx} % the \si environment for units
\usepackage{mathtools} % The dcases environment, prettier than just cases
\usepackage{tikz} % For drawing picures

\title{Exercise 6 solutions - TFY4345 Classical Mechanics}
\date{2020}

\begin{document}
    \maketitle

    \section{Elastic scattering in laboratory coordinates}
    The relations we start with are

    \begin{equation*}
        \cos(\vartheta) = \frac{\cos(\Theta) + \rho}{\sqrt{1 + 2\rho \cos(\Theta) + \rho^2}}, \quad \sigma'(\vartheta) = \sigma (\Theta) \frac{(1 + 2\rho \cos(\Theta) + \rho^2)^{3/2}}{1 + \rho \cos(\Theta)}, \quad \rho = m_1 / m_2.
    \end{equation*}
    By the assumption of equal masses, we get $\rho = 1$, and thus
    \begin{equation*}
        \cos(\vartheta) = \frac{1 + \cos(\Theta)}{\sqrt 2 \sqrt{1 + \cos(\Theta)}} = \sqrt{\frac{1 + \cos(\Theta)}{2}}.
    \end{equation*}
    Using the provided trigonometric identity, 
    \begin{equation*}
        \cos(\vartheta) = \cos(\Theta / 2) \implies \vartheta = \Theta/2
    \end{equation*}
    This means scattering angels of above $90^\circ$ is not possible in the lab system.
    \\
    \\
    The relation for the cross sections becomes
    \begin{equation*}
        \sigma'(\vartheta) = \sigma(\Theta) \frac{2^{3/2}(1 + \cos(\Theta))^{3/2}}{1 + \cos(\Theta)} = 4 \sqrt{\frac{ 1 + \cos(\Theta)}{2}} \sigma(\Theta) = 4 \cos(\Theta/2) \sigma(\Theta), \quad \vartheta \leq \pi/2
    \end{equation*}
    This means that for isotropic scattering (when the cross section in the CM frame $\sigma(\Theta)$ is constant), the cross section in the lab frame goes as the cosine of the scattering angle, also in the lab frame.
    \\
    \\
    The velocity of of particle $i$ after the collision is denoted by $v_i$ in the lab frame, and $v_i'$ in the CM frame, while the velocity of the CM frame in the lab frame is denote by $V$. Assume $i=1$ was the incident particle. We then have the relation (see lecture notes, chapter 4)
    \begin{equation*}
        v_1^2 = v_1'^2 + V^2 + 2v_1'V \cos(\Theta), \quad V = \frac{\mu}{m_2} v_0, \quad \rho = \frac{\mu}{m_2}\frac{v_0}{v_1'}
    \end{equation*}
    where $\mu = m_1m_2 / (m_1 + m_2)$ is the reduced mass, while $v_0$ is the initial velocity of the incident particle. Inserting $m_1 = m_2$, we get
    \begin{equation*}
        V = \frac{1}{2}v_0, \quad v_1' = \frac{1}{2}v_0
    \end{equation*}
    This then gives
    \begin{equation*}
        v_1^2 = 2(\frac{1}{2}v_0)^2 + 2 (\frac{1}{2}v_0)^2 \cos(\Theta) = \frac{1}{2}v_0 (1 + \cos(\Theta)).
    \end{equation*}
        The relation for the kinetic energy before and after the collision is therefore,
    \begin{equation*}
        \frac{E_1}{E_0} = \frac{v_1^2}{v_o^2} = \frac{1 + \cos(\Theta)}{2} = \cos^2(\vartheta)
    \end{equation*}

    \section{Rotating system in cylindrical coordinates.}
    Cylindrical coordinates are given by the relations
    \begin{equation*}
        x = r \cos(\theta), \quad y = r \sin(\theta), z = z,
    \end{equation*}
    giving
    \begin{flalign*}
        \hspace{3cm}
        &
        \begin{dcases}
            \dot x^2 = \dot r \cos(\theta) - r \dot \theta \sin(\theta) \\
            \dot y^2 = \dot r \sin(\theta) + r \dot \theta \sin(\theta) \\
            \dot z^2 = \dot z^2
        \end{dcases}
        &
    \end{flalign*}
    The kinetic energy of a particle of mass $m$ is then
    \begin{align}
        T &= \frac{1}{2}m(\dot x^2 + \dot y^2 + \dot z^2) 
        = \frac{1}{2}m \bigg[ 
            \dot r^2 \big(\cos^2 (\theta) + \sin^2(\theta)\big) 
            + 2\dot r r \dot \theta \big(\cos(\theta)\sin(\theta) - \cos(\theta)\sin(\theta)\big) \\
        & + r^2 \dot \theta\big(\cos^2 (\theta) + \sin^2(\theta)\big) +  \dot z^2\bigg]  = \frac{1}{2}(\dot r^2 + (r\dot \theta)^2 + \dot z^2),
    \end{align}
    giving the lagrangian
    \begin{equation*}
        L = T - V = \frac{1}{2} m [\dot r^2 + (r\dot \theta)^2 + \dot z^2] - V(r, \theta, z).
    \end{equation*}
    The needed derivatives are
    \begin{flalign*}
        \hspace{3cm}
        &        
        \begin{dcases}
            r: \quad \pdv r L = m r\dot \theta ^2 - \pdv{V}{r}, \quad \dv t \pdv{\dot r} L = \dv t m \dot r = m \ddot r \\
            \theta: \quad \pdv \theta L =  -\pdv{V}{\theta}, \quad \dv{t} \pdv{\dot \theta} L = \dv t (m r^2 \dot \theta) = 2m\dot r r \dot \theta + m r^2 \ddot \theta \\
            z: \quad \pdv{L}{z} = -\pdv{V}{z}, \quad \dv t \pdv{\dot z} L = \dv t m \dot z = m \ddot z,
        \end{dcases}
        &
    \end{flalign*}
    which gives the equations of motion
    \begin{flalign*}
        \hspace{3cm}
        &
        \begin{dcases}
            m \ddot r = m r\dot \theta ^2 - \pdv{V}{r} \\
            2m\dot r r \dot \theta + m r^2 \ddot \theta = -\pdv{V}{\theta} \\
            m \ddot z = -\pdv{V}{z}.
        \end{dcases}
        &
    \end{flalign*}

    The canonical momenta can be read of the earlier derivatives, and are
    \begin{flalign*}
        \hspace{3cm}
        &
        \begin{dcases}
            p_r = m \dot r \\
            p_\theta = m r^2 \dot \theta \\
            p_z = m \dot z.
        \end{dcases}
        &
    \end{flalign*}
    This means that we can rewrite the kinetic energy term as
    \begin{equation*}
        T = \frac{1}{2} \bigg[\frac{p_r^2}{m} + \frac{p_\theta^2}{r^2m} + \frac{p_z^2}{m}\bigg]
    \end{equation*}
    As the assumptions of a time independent Lagrangian, a conservative force and a kinetic energy which is quadratic in the velocity terms (see Goldstein p. 338-339), we can write 
    \begin{equation*}
        H = T + V =  \frac{1}{2} \bigg[\frac{p_r^2}{m} + \frac{p_\theta^2}{r^2m} + \frac{p_z^2}{m}\bigg] + V(r, \theta, z)
    \end{equation*}
    Hamiltons equations of motion are then
    \begin{flalign*}
        \hspace{3cm}
        &
        \begin{dcases}
            \dot q_r =  \pdv{H}{p_r} = \frac{p_r}{m}, 
            \quad -\dot p_r = \pdv{H}{r} = \pdv{V}{r}  - \frac{p_\theta^2}{r^3 m} \\
            \dot q_\theta =  \pdv{H}{p_\theta} = \frac{p_\theta}{r^2m}, 
            \quad -\dot p_\theta = \pdv{H}{\theta} = \pdv{V}{\theta}\\
            \dot q_z =  \pdv{H}{p_z} = \frac{p_z}{m}, \quad -\dot p_z = \pdv{H}{z} = \pdv{V}{z} \\
        \end{dcases}
        &
    \end{flalign*}

    \section{Centrifugal force and gravitation}
    In the rotating coordinate system, the centrifugal force acting on an element of length $\dd r$, a distance $r$ from the center, is
    \begin{equation*}
        \dd F_c = r \omega^2 \lambda \, \dd r.
    \end{equation*}
    The gravitational pull on the same piece of rod is given by 
    \begin{equation*}
        \dd F_g = \frac{G m \lambda \dd r}{r^2} = \frac{g R^2 \lambda \dd r}{r^2},
    \end{equation*}
    where $g$ is the gravitational acceleration on the surface of the earth. Balancing the centrifugal and gravitational force then amounts to demanding that
    \begin{align}
        & F_c = \int_{R}^{R + L} r \omega^2 \lambda \, \dd r = \omega^2 \rho \lambda (2RL + L^2) \\ 
        = & F_g = \int_{R}^{R + L} \frac{g R^2 \lambda \dd r}{r^2} = 2g_0 R^2 \lambda \frac{L}{(R + L)R} \\
        \implies &L^2 + 3RL  + \bigg(2R^2- \frac{2g_0 R }{\omega^2 } \bigg) = 0 \implies L = -\frac{3 R}{2} + \frac{1}{2}\sqrt{R^2 + \frac{8 g R }{\omega^2 }}.
    \end{align}
    Setting values for the earth, $\omega  = 2 \pi /(1 \mathrm{day}), \, R = 6.4 \cdot 10^3 \si{km}, \, g = 9.8 \si{m.s^{-1}}$, we get $L = 1.4 \cdot 10^5 \si{km}$, or $1/e$ of the way to the moon.

    \section{Coriolis effect on a falling particle}
    We do not need to take into account the centrifugal force, as it is proportional to $\omega^2$ and thus negligible. The acceleration seen in the coordinate system rotating with the earth is then (see chapter 5.H in the compendium, chapter 4.10 in Goldstein, p. 174 - 180)
    \begin{equation*}
        \mathbf{a_s} = \mathbf{a_r} + 2 \mathbf{\omega} \cross \mathbf{v_r}
    \end{equation*}
    When taking the cross product, the coriolis force is negligible compared to the acceleration due to the gravity of the earth ($\omega |\mathbf{v_r}| \ll g $), and it is thus a very good first order approximation that the motion of the particle is
    \begin{equation*}
        \dot x = 0, \, \dot y = 0, \, \dot z = -gt.
    \end{equation*}
    We then have
    \begin{equation*}
        \omega \cross \mathbf{v}_r= 
        \begin{vmatrix}
            \mathbf{e}_x & \mathbf{e}_x & \mathbf{e}_x \\
            -\omega \cos(\alpha) & 0 & \omega \sin(\alpha) \\
            0 & 0 & -g t
        \end{vmatrix}
        = -\omega g t \cos(\alpha) \mathbf{e}_y
    \end{equation*}
    The acceleration in the non-rotating frame of reference is
    \begin{equation*}
        \mathbf{a_s} = -g \mathbf{e_z}
    \end{equation*}
    The equations of motion in the rotating frame of reference thus become
    \begin{equation*}
        (\mathbf{a}_r)_x = \ddot x = 0, \quad (\mathbf{a}_r)_y = \ddot y = 2 \omega g t \cos(\alpha), \quad (\mathbf{a_r})_z = \ddot z = -g  
    \end{equation*}
    Integrating twice, and setting $y(0) = \dot y(0) = \dot z(0) = 0$ then gives the motion due to the coriolis effect,
    \begin{equation*}
        y(t) = \frac{1}{3}\omega g \cos(\alpha) t^3, \quad z(t) = z_0 - \frac{1}{2}g t^2
    \end{equation*}
    The fall time from the height $h = z_0$ is $t = \sqrt{2h / g}$, so the total eastward deflection is
    \begin{equation*}
        d = \frac{1}{3} \omega \cos(\alpha)\sqrt{\frac{8 h^3}{g}}.
    \end{equation*}
    An object dropped from $100 \si{m}$, at latitude $45^\circ $ north is deflected approximately $1.55 \si{cm}$. 





\end{document}
