\documentclass{article}

\usepackage{physics} % Handy shortcuts like \pdv, \dd and much more
\usepackage{geometry} % smaller margins, can be adjusted if given arguments
\usepackage{siunitx} % the \si environment for units
\usepackage{mathtools} % The dcases environment, prettier than just cases
\usepackage{tikz} % For drawing picures
\usepackage{wrapfig} % Wrapping text around figures


\title{Exercise 11 solutions - TFY4345 Classical Mechanics}
\date{2020}

\begin{document}
    \maketitle
    \section{Binding energy of the deuteron}
        Deuteron can be split by gamma rays in the nuclear reaction $\gamma + {^2\mathrm{H}} \rightarrow p + n$. Let us assume that the kinetic energy of the gamma ray is transferred \emph{completely} to the rest mass energy of the proton and the neutron. This way, we get the minimum energy required. We have
        \begin{itemize}
            \item Mass of a proton ($^1\mathrm{H} = p$): $1.007825 \si{u}$
            \item Mass of a neutron ($n$): $1.008665 \si{u}$
            \item Sum ($p + n$): $2.016490 \si{u}$
            \item Mass of deuteron ($^2 \mathrm{H}$): $2.014102$
        \end{itemize}
        The energy needed by the gamma-ray is then the difference in energy which holds the proton and neutron together to form a deuteron, $Q=0.00238 \si{u}$. The conversion factor between atomic units and electron volt is $1 \si{u} = 931.5 \si{MeV c^{-2}}$. This means the binding energy of the deuteron is $E = mc^2 = (0.002388 \mathrm{u} \times 931.5 \mathrm{MeV c^{-2} u^{-1}})c^2 = 2.224 \si{MeV}$. Correspondingly, the energy of the initial gamma ray has to be \emph{at least} $2.22\si{MeV}$ (threshold energy) to split a deuteron apart.

    \section{Frequency shift on a rotating disk}
        Let $\dd N$ be the number of wave lengths emitted from the $^{57}\mathrm{Co}$ in the time interval $\dd t'$, in the system $S'$ in which the atom is instantaneous at rest. $\dd t' = \dd \tau$ is the eigentime of $S'$. The frequency as measured in $S'$ is
        \begin{equation*}
            f_0 = \dv{N}{t'}.
        \end{equation*}
        If $t$ is the time as measured by the observer at the center, we have $\dd t = \gamma \dd \tau$. The frequency in the center is then
        \begin{equation}
            f = \dv{N}{t} = \frac{1}{\gamma} \dv{N}{\tau} = \frac{f_0}{\gamma} = \sqrt{1 - \beta^2}f_0
        \end{equation}
        This is the transversal doppler effect, and is only due to time dilation, and not length contraction. Special relativity is sometimes described as only being valid when no acceleration is involved, but this is not true. 

    \section{Fast moving particle in two inertial frames}
        The Lorentz transformation gives the relation between the coordinates in $S$ and $S'$:
        \begin{align*}
            x' & = x \\
            y' & = y \\
            z' & = \gamma (z - vt) \\
            t' & = \gamma \left(t - \frac{v z}{c^2}\right), \quad \gamma = \frac{1}{\sqrt{1 - (v/c)^2}}
        \end{align*}
        The four velocity of an object is the derivative of the spatial coordinates with respect to the time coordinate in that system. Therefore, using the relations we found and the chain rule, 
        \begin{align*}
            &u_x' = \dv{x'}{t'} = \dv{x}{t'} = \dv{x}{t} \frac{1}{\dv*{t'}{t}}, \quad
            \dv{t'}{t} = \dv{t} \bigg( \gamma \left(t - \frac{v z}{c^2}\right)\bigg) = \gamma \left(1 - \frac{v u_z}{c^2}\right) \\
            &\implies u_x' = \frac{u_x}{\gamma (t - v u_z/c^2)}.
        \end{align*}
        In the same way,
        \begin{equation*}
            u_y' = \frac{u_y}{\gamma (t - v u_z/c^2)}.
        \end{equation*}
        The $z$ component is given by
        \begin{align*}
            &u_z' = \dv{z'}{t'} = \dv{z'}{t} \frac{1}{\dv*{t'}{t}}, \quad \dv{z'}{t} = \dv{t} \left(\gamma (z - vt) \right) = \gamma (u_z - v) \\
            & \implies u_z' = \frac{u_z - v}{t - v u_z /c^2}
        \end{align*}
        This is the general version of Einstein's velocity addition, where the $S''$ system (the particle) is not necessarily moving in the $z$-direction. A quick, less formal way of deriving these formula is taking the differential:
        \begin{align*}
            & \dv{x'}{t'} = \frac{\dd z}{\gamma(\dd t - v (\dd z)/c^2)} = \frac{\dv*{x}{t}}{\gamma (\dv*{t}{t} - v \dv*{z}{t}/c^2)} = \frac{u_x}{\gamma (t - v u_z/c^2)} \\
            & \dv{z'}{t'} = \frac{\gamma(\dd z - v \dd t)}{\gamma (\dd t - v \dd z/c^2)} = \frac{\gamma(\dv*{z}{t} - v \dv*{t}{t})}{\gamma (\dv*{t}{t} - v (\dv*{z}{t})/c^2)} = \frac{u_z - v}{t - v u_z /c^2}
        \end{align*}

    \section{Lorentz transformation of energy and momentum}
        a) Using Einstein's addition formula, 
        \begin{align*}
            u'^2 &= u_x'^2 + u_y'^2 + u_z'^2 = \left(\frac{u_x}{\gamma(1 - v u_z /c^2)}\right)^2 + \left(\frac{u_y}{\gamma(1 - v u_z /c^2)}\right)^2 + \left(\frac{u_z - v}{1 - v u_z /c^2}\right)^2\\
            & = \frac{u_x^2 + u_y^2 + \gamma^2(v - u_z)^2}{\gamma^2(1 - v u_z/c^2)^2} \\
        \end{align*}
        We use this to show
        \begin{align*}
            & 1 - \frac{u'^2}{c^2} = \left[\frac{1}{\gamma(1 - v u_z /c^2)}\right]^2 \left( \gamma^2\left(1 - \frac{u_z v}{c^2}\right)^2- \frac{u_x^2 + u_y ^2}{c^2} - \gamma^2 \frac{(u_z - v)^2}{c^2}\right) \\
            & = \left[\frac{1}{\gamma(1 - v u_z /c^2)}\right]^2 
            \left( \gamma^2\left(1 - 2\frac{u_z v}{c^2} + \frac{u_z^2 v^2}{c^4} - \frac{u_z^2 - 2u_zv + v^2}{c^2}\right) - \frac{u_x^2 + u_y ^2}{c^2} \right) \\
            & = \left[\frac{1}{\gamma(1 - v u_z /c^2)}\right]^2 
            \left( \gamma^2\left(1 + \frac{u_z^2 v^2}{c^4} - \frac{u_z^2 + v^2}{c^2}\right) - \frac{u_x^2 + u_y^2}{c^2} \right) \\
            & = \left[\frac{1}{\gamma(1 - v u_z /c^2)}\right]^2 
            \left( \gamma^2\left(1- \frac{v^2}{c^2}  + \left[\frac{v^2}{c^2} -1\right]\frac{u_z^2}{c^2} \right) - \frac{u_x^2 + u_y^2}{c^2} \right) \\
            & = \left[\frac{1}{\gamma(1 - v u_z /c^2)}\right]^2
            \left( \gamma^2\left(1- \frac{v^2}{c^2} \right)\left(  1-\frac{u_z^2}{c^2} \right) - \frac{u_x^2 + u_y^2}{c^2} \right) \\
            & = \left[\frac{1}{\gamma(1 - v u_z /c^2)}\right]^2
            \left( 1 - \frac{u_x^2 + u_y^2 + u_z^2}{c^2} \right) = \frac{1 - u^2/c^2}{\gamma^2(1 - v u_z /c^2)^2}.
        \end{align*}
        This gives the desired relation
        \begin{equation*}
            \frac{1}{\sqrt{1 - (u'/c)^2}} = \gamma \frac{1 - v u_z/c^2}{\sqrt{1 - (u/c)^2}}.
        \end{equation*}
        b) We can now use this relation to find the transformation rule for energy. Starting with the equation given,
        \begin{equation*}
            E' = \frac{mc^2}{\sqrt{1 - (u'/c)^2}} = mc^2 \gamma \frac{1 - v u_z/c^2}{\sqrt{1 - (u/c)^2}} =  \gamma \frac{mc^2}{\sqrt{1 - (u/c)^2}} - \gamma \frac{m v u_z}{\sqrt{1 - (u/c)^2}}
        \end{equation*}
        Recognizing the formula for $E$ and $p_z$, this gives the relation
        \begin{equation*}
            E' = \gamma(E - vp_z),
        \end{equation*}
        For the $x$-component of momentum, we get
        \begin{equation*}
            p_x'  =\frac{mu'_x}{\sqrt{1 - u'/c^2}} = m u_x' \gamma \frac{1 - v u_z/c^2}{\sqrt{1 - (u/c)^2}}.
        \end{equation*}
        Einstein's addition formula gives $u_x' \gamma(1 - v u_z/c^2) = u_x$, so
        \begin{equation*}
            p_x' = m \frac{u_x}{\sqrt{1 - (u/c)^2}} = p_x.
        \end{equation*}
        In the same way, we get 
        \begin{equation*}
            p_y' = p_y.
        \end{equation*}
        Finally, the Einstein's formula for the $z$-component of the velocity gives $u_z - v = u_z'(1 - v u_z/ c^2)$, so
        \begin{equation*}
            p_z' =  m u_z' \gamma \frac{1 - v u_z/c^2}{\sqrt{1 - (u/c)^2}} =\frac{m\gamma(u_z - v)}{\sqrt{1 - (u/c)^2}} = \gamma(p_z - vE/c^2)
        \end{equation*}


\end{document}
