\documentclass{article}

\usepackage{physics} % Handy shortcuts like \pdv, \dd and much more
\usepackage{geometry} % smaller margins, can be adjusted if given arguments
\usepackage{siunitx} % the \si environment for units
\usepackage{mathtools} % The dcases environment, prettier than just cases
\usepackage{tikz} % For drawing picures
\usepackage{wrapfig} % Wrapping text around figures
\usepackage{cancel}

\title{Exercise 3 solutions - TFY4345 Classical Mechanics}
\date{2020}

\begin{document}
    \maketitle
    \section{Pendulum on spinning a wheel}
        With the origin of our coordinate system in the center of the rotating rim, the Cartesian components of the mass $m$ become
        \begin{align*}
            x = a \cos(\omega t) + b \sin(\theta) \\
            y = a \sin(\omega t) - b \cos(\theta).
        \end{align*}
        The velocities are
        \begin{align*}
            \dot x = - a \omega \sin(\omega t) + b \dot \theta \cos(\theta) \\
            \dot y = a \omega \cos(\omega t) + b \dot \theta \sin(\theta).
        \end{align*}
        Taking the time derivative once again gives the acceleration:
        \begin{align*}
            \ddot x = -a \omega^2 \cos(\omega t) + b(\ddot \theta \cos(\theta) - \dot \theta^2 \sin(\theta)) \\
            \ddot y = - a \omega^2 \sin(\omega t) + b(\ddot \theta \sin(\theta) + \dot \theta^2 \cos(\theta)).
        \end{align*}
        It should be clear that the single generalize coordinate is $\theta$. The kinetic and potential energies are 
        \begin{equation*}
            T = \frac{1}{2}m (\dot x^2 + \dot y^2), \, V = mgy.
        \end{equation*}
        Inserting what we found earlier, the Lagrangian becomes
        \begin{equation*}
            L = \frac{1}{2}m[a^2 \omega^2 + b \dot \theta^2 + 2 b \theta^2 a \omega \sin(\theta - \omega t)] - mg[a \sin(\omega t) - b \cos(\theta)].
        \end{equation*}
        The derivatives needed for the equation of motion are
        \begin{align*}
            \dv{t}\pdv{L}{\dot \theta} = mb^2 \ddot \theta + m b a \omega (\dot \theta - \omega) \cos(\theta - \omega t), \\
            \pdv{L}{\theta} = mba \dot \theta \omega \cos(\theta - \omega t) - m g b \sin(\theta).
        \end{align*}
        Inserting this into Euler's equation, and solving for $\ddot \theta$ gives
        \begin{equation*}
            \ddot \theta = \frac{\omega^2 a}{b}\cos(\theta-\omega t) - \frac{g}{b} \sin(\theta).
        \end{equation*}
        Notice that, for $\omega = 0$, this reduces to the equation for the simple pendulum.

    \section{Bead on a ring}
        (FIGUR) \\ \\
        The potential energy is given by
        \begin{equation*}
            U = mgh = mgR(1 - \cos(\theta)),
        \end{equation*}
        while the kinteic energy is
        \begin{equation*}
            T = \frac{1}{2}m v^2 = \frac{1}{2}m \left((r \dot \varphi)^2 + (R \dot \theta)\right) = \frac{1}{2}m R^2\left(\sin^2(\theta) \dot \varphi^2 + \dot\theta^2\right).
        \end{equation*}
        The Euler equation for $\varphi$ is given by
        \begin{equation*}
            \pdv{L}{\varphi} = 0 \implies \dv{t}\pdv{L}{\dot \varphi} = \dv{t}\left(m \sin(\theta)R^2 \dot \varphi\right) = 0 \implies \dot \varphi = \omega = \mathrm{const.}
        \end{equation*}
        The equation for $\theta$ is given by
        \begin{align*}
            &\pdv{L}{\theta} = m R^2 \cos(\theta) \sin(\theta) \dot \varphi^2 - mgR \sin(\theta), \, \dv{t} \pdv{L}{\dot \theta} = m R^2 \ddot \theta, \\
            & \implies \ddot \theta = R \cos(\theta) \sin(\theta) \dot \varphi^2  - g \sin(\theta)
        \end{align*}
        In the equilibrium position, we have that $\ddot \theta = 0$. This means
        \begin{equation*}
            \cos(\theta) = \frac{g }{R\dot \varphi^2} = \frac{g }{R\omega^2}.
        \end{equation*}
        Note: we could have set $\dot \varphi = 0$ right at the beginning, and treat $\theta$ as the sole generalized variable.

    \section{Atwood's machine}
        The angular velocity of the pulley is 
        \begin{equation*}
            \omega = \frac{\dot x_2}{a}.
        \end{equation*}
        This means the kinetic energy of the system is 
        \begin{equation*}
            T = \frac{1}{2}m_1\dot x_1^2+\frac{1}{2}m_2\dot x_2^2 + \frac{1}{2}I \omega^2
        \end{equation*}
        The length of the string is a constant, so 
        \begin{equation*}
            x_1 + x_2 = \ell = \mathrm{const.}
        \end{equation*}
        Inserting this into the kinetic energy gives 
        \begin{equation*}
            T = \frac{1}{2}(m_2-m_1)\dot x_1^2+\frac{1}{2}m_2\dot x_2^2 + \frac{1}{2}I \left(\frac{\dot x}{a}\right)^2.
        \end{equation*}
        The potential energy is
        \begin{equation*}
            V = - m_1 g x_1 - m_2 g x_2 = -m_1g(\ell -x_2) - m_2g(x_2),
        \end{equation*}
        so the Lagrangian is
        \begin{equation*}
            L = \frac{1}{2}\left(m_1 + m_2 + \frac{I}{a^2}\right) \dot x_2^2 + m_1g(\ell - x_2) + m_2 g x_2.
        \end{equation*}
        The canonical momentum is
        \begin{equation*}
            p_{2} = \pdv{L}{\dot x^2} = \left(m_1 + m_2 + \frac{I}{a^2}\right) \dot x_ 2
        \end{equation*}
        The conditions are met so that we can write the Hamiltonian as
        \begin{equation*}
            H = T + v = \frac{1}{2}\frac{p_2^2}{m_1 + m_2 + I/a^2} -m_1g(\ell - x_2) - m_2 g x_2.
        \end{equation*}
        Hamilton's equations then become
        \begin{align*}
            & \dot x_2 = \pdv{H}{p_2} = \frac{p_2}{m_1 + m_2 + I/a^2} \\
            & \dot p_2 = - \pdv{H}{x_2} = (m_2 - m_1)g.
        \end{align*}
        Lastly, $H$ is conserved as
        \begin{equation*}
            \dv{H}{t} = -\pdv{L}{t} = 0.
        \end{equation*}


    \section{Particle on a moving wedge}
        (FIGUR) \\ \\
        The position of the mass $m$ in a coordinate system moving with the wedge is
        \begin{equation*}
            \mathbf{r}' = r \cos(\theta) \hat e_x + r \sin(\theta) \hat e_y.
        \end{equation*}
        In the laboratory coordinates, which does not move with the wedge, the wedge has position $x \hat e_x$, so the position of the mass $m$ is 
        \begin{equation*}
            \mathbf{r} = (x + r \cos(\theta)) \hat e_x + \sin(\theta) \hat e_y.
        \end{equation*}
        The velocity is
        \begin{equation*}
            \mathbf{\dot r}
            = (\dot x  + \dot r \sin(\theta) - r \dot \theta \sin(\theta)) \hat e_x
            + (\dot r \sin(\theta) + r \dot \theta \cos(\theta)) \hat e_y,
        \end{equation*}
        while the square velocity becomes
        \begin{align*}
            \dot r^2 = & \left(\dot x  + \dot r \cos(\theta) - r \dot \theta \sin(\theta)\right)^2 + \left(\dot r \sin(\theta) + r \dot \theta \cos(\theta)\right)^2 = \\
            & \dot x^2 + 2 \dot x \left(\dot r \cos(\theta) - r \dot \theta \sin(\theta)\right) + \left(\dot r \cos(\theta) - r \dot \theta \sin(\theta)\right)^2 
            + (\dot r \cos(\theta))^2 + 2r \dot r \dot \theta \cos(\theta) \sin(\theta) + (r \dot \theta \sin(\theta))^2\\
            & = \dot x^2 + \underline{(\dot r\cos(\theta))^2} + \underline{(r \dot \theta \sin(\theta))^2} + 2 \dot x \dot r \cos(\theta) - 2 \dot x r \dot \theta \sin(\theta) - \cancel{2 r \dot r \dot \theta \cos(\theta) \sin(\theta)} \\
            & + \underline{(\dot r \sin(\theta))^2} + \cancel{2r \dot r \dot \theta \cos(\theta) \sin(\theta)} + \underline{(r \dot \theta \cos(\theta))^2}\\
            & = \dot x^2 + \dot r^2 + (r \dot \theta)^2+ 2 \dot x \dot r \cos(\theta) - 2 \dot x r \dot \theta \sin(\theta). 
        \end{align*}
        The potential energy is given by
        \begin{equation*}
            V = -m g r \sin(\theta).
        \end{equation*}
        The restriction of the little mass to stay on the wedge is given by $r - R = 0$, so the total Lagrangian, including the undetermined multiplier becomes
        \begin{equation*}
            L = \frac{1}{2}m \left(\dot x^2 + \dot r^2 + (r \dot \theta)^2+ 2 \dot x \dot r \cos(\theta) - 2 \dot x r \dot \theta \sin(\theta) \right) + \frac{1}{2}M \dot x^2 + mgr^2\sin(\theta) + \lambda(r - R).
        \end{equation*}

\end{document}

