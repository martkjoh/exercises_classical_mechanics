\documentclass{article}

\usepackage{physics} % Handy shortcuts like \pdv, \dd and much more
\usepackage{geometry} % smaller margins, can be adjusted if given arguments
\usepackage{siunitx} % the \si environment for units
\usepackage{mathtools} % The dcases environment, prettier than just cases
\usepackage{tikz} % For drawing picures
\usepackage{wrapfig} % Wrapping text around figures
\usepackage{cancel} % Strikekethrough parts of equations


\title{Exercise 12 - TFY4345 Classical Mechanics}
\date{2020}

\begin{document}
    \maketitle
    \section{Generating function $F_4$}
    The generating function $F$ is given by
    \begin{equation*}
        F = q_ip_i - Q_iP_i + F_4(p, P, t).
    \end{equation*}
    This means the time derivative can be written as
    \begin{equation*}
        \dv{F}{t} = \dot p_i q_i + p_i \dot q_i - \dot P_i Q_i - P_i \dot Q_i + \dv{F(p, P, t)}{t}.
    \end{equation*}
    Inserting this into the relation between the original Hamiltonian $H$ and the new one, $K$
    \begin{equation*}
        p_i \dot q_i - H(q, p, t) = P_i \dot Q_i - K(Q, P, t) + \dv{F}{t}
    \end{equation*}
    gives
    \begin{align*}
        \cancel{p_i \dot q_i} - H(q, p, t) = \cancel{P_i \dot Q_i} - K(Q, P, t) + \dot p_i q_i + \cancel{p_i \dot q_i} - \dot P_i Q_i - \cancel{P_i \dot Q_i} + \dv{F(p, P, t)}{t} \\
        \dot p_i q_i  + H(q, p, t) =  \dot P_i Q_i + K(Q, P, t) - \dv{F_4(p, P, t)}{t} 
    \end{align*}
    We can expand 
    \begin{equation*}
        \dv{F_4(p, P, t)}{t} = \pdv{F_4}{t} + \pdv{F_4}{p_i} \dot p_i +  \pdv{F_4}{P_i} \dot P_i,
    \end{equation*}
    which gives
    \begin{equation*}
        \dot p_i q_i  + H(q, p, t) =  \dot P_i Q_i + K(Q, P, t) - \left(\pdv{F_4}{t} + \pdv{F_4}{p_i} \dot p_i +  \pdv{F_4}{P_i} \dot P_i\right).
    \end{equation*}
    This only holds if
    \begin{equation*}
        K = H + \pdv{F_4}{t}, \quad q_i = - \pdv{F_4}{p_i}, \quad Q_i = \pdv{F_4}{P_i},
    \end{equation*}
    which is the equations we were looking
        
    \section{The Poisson bracket}
    The Hamiltonian for the harmonic oscillator is
    \begin{equation*}
        H = \frac{1}{2m} \left(p^2 + m^2 \omega^2 q^2\right),
    \end{equation*}
    and we have the canonical transformations
    \begin{equation*}
        q = \sqrt{\frac{2P}{m \omega}} \sin(Q), \quad p = \sqrt{2 P m \omega} \cos(Q), \quad H = \omega P.
    \end{equation*}
    The Poisson bracket in the original is
    \begin{equation*}
        [q, H]_{q, p} = \underbrace{\pdv{q}{q}}_{=1}\pdv{H}{p} - \underbrace{\pdv{q}{q}}_{=0}\pdv{H}{p} = \pdv{H}{p} = \frac{p}{m}.
    \end{equation*} 
    In the new coordinates the bracket is
    \begin{equation*}
        [q, H]_{Q, P} = \pdv{q}{Q}\pdv{H}{P} - \pdv{q}{P} \underbrace{\pdv{H}{Q}}_{= 0} = \pdv{q}{Q}\pdv{H}{P},
    \end{equation*}
    where
    \begin{align*}
        & \pdv{q}{Q} = \pdv{Q}\left(\sqrt{\frac{2P}{m \omega}} \sin(Q)\right) = \sqrt{\frac{2P}{m \omega}} \cos(Q) \cdot \frac{\sin(Q)}{\sin(Q)} = q \cot(Q), \\
        & \pdv{H}{P} = \omega, \quad \cot(Q) = \frac{\cos(Q)}{\sin(Q)} = m \omega \frac{p}{q}.
    \end{align*}
    This gives 
    \begin{equation*}
        [q, H]_{Q, P} = \omega q \cot{Q} = \frac{p}{m} = [q, H]_{q, p}.
    \end{equation*}
    The fact that $[q, H] \neq 0$ means that $q$ is not a constant of motion.
    \section{The symplectic condition}
 
    \section{Free particle}
 
\end{document}
