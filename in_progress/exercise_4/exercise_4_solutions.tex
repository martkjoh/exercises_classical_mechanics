\documentclass{article}

\usepackage{physics} % Handy shortcuts like \pdv, \dd and much more
\usepackage{geometry} % smaller margins, can be adjusted if given arguments
\usepackage{siunitx} % the \si environment for units
\usepackage{mathtools} % The dcases environment, prettier than just cases
\usepackage{tikz} % For drawing picures
\usepackage{wrapfig} % Wrapping text around figures
\usepackage[normalem]{ulem} % Dotted / dashed underline
\usepackage{cancel}

\title{Exercise 4 solutions - TFY4345 Classical Mechanics}
\date{2020}

\begin{document}
    \maketitle
    \section{Mathematical pendulum}
        The position of the mass is
        \begin{align*}
            x = \ell \sin(\theta), \quad y = \frac{1}{2} a t^2  - \ell \cos(\theta) \\
            \dot x = \ell \dot \theta \cos(\theta) \quad y = a t + \ell \dot \theta \sin(\theta),
        \end{align*}
        so the kinetic energy is given by
        \begin{equation*}
            T = \frac{1}{2}m \left(\dot x^2 + \dot y^2\right) = \frac{1}{2}m \left( (\ell \dot \theta)^2  + (a t)^2 + 2 a t \ell \dot \theta \sin(\theta)\right),
        \end{equation*}
        and the potential energy is 
        \begin{equation*}
            V = mgy = mg \left(\frac{1}{2}a t^2 - \ell \cos(\theta)\right).
        \end{equation*}
        The Lagrangian is
        \begin{equation*}
            L = \frac{1}{2}m \left( (\ell \dot \theta)^2  + (a t)^2 + 2 a t \ell \dot \theta \sin(\theta)\right) - mg \left(\frac{1}{2}a t^2 - \ell \cos(\theta)\right),
        \end{equation*}
        so the canonical momentum is
        \begin{equation*}
            p_\theta = \pdv{L}{\dot \theta} = m\left( \ell^2 \dot \theta + a t \ell\sin(\theta)\right) \implies 
            \dot \theta  = \frac{p_\theta - m t a \ell\sin(\theta)}{m \ell^2}.
        \end{equation*} 
        This gives the Hamiltonian
        \begin{align*}
            H &= \dot \theta p_\theta - L \\
            & = p_\theta \frac{p_\theta - m t a \ell \sin(\theta)}{m \ell^2} - 
            \frac{1}{2}m \left[ \ell^2 \left(\frac{p_\theta - m t a \ell\sin(\theta)}{m \ell^2}\right)^2  + (\ell at)^2 + 2 a t \ell \sin(\theta) \left(\frac{p_\theta - m t a \ell \sin(\theta)}{m \ell^2}\right) \right] \\
            & +  mg \left(\frac{1}{2}a t^2 - \ell \cos(\theta)\right) \\
        \end{align*}
        \begin{align*}
            & = \frac{1}{m \ell^2} \left(p_\theta^2 - p_\theta m t a \ell \sin(\theta)\right) - \\
            & \frac{1}{2 m \ell^2} \left[ p_\theta^2 - 2 p_\theta mta\ell \sin(\theta) + (mta\ell \sin(\theta))^2 + (m\ell ta)^2 + 2 p_\theta m a t \ell \sin(\theta) - 2 (m t a \ell \sin(\theta))^2 \right] \\
            & +  mg \left(\frac{1}{2}a t^2 - \ell \cos(\theta)\right) \\
            & = \frac{1}{2 m \ell^2} \left(p_\theta - m t a \ell \sin(\theta)\right)^2 - \frac{1}{2} m a^2 t^2 + \frac{1}{2}m a g t^2 - mg \ell \cos(\theta).
        \end{align*}
        The Hamiltonian equations of motion
        \begin{align*}
            & \dot \theta = \pdv{H}{p_\theta} = \frac{p_\theta - m t a \ell\sin(\theta)}{m \ell^2} \\
            & \dot p_\theta = - \pdv{H}{\theta} = \frac{a t \cos(\theta)}{\ell} \left[ p_\theta - m a t \ell \sin(\theta) \right] - m g \ell \sin(\theta).
        \end{align*}
        Furthermore, we see that $H \neq T + V$, so the Hamiltonian function is not the total energy of the system. Furthermore, 
        \begin{equation*}
            \dv{H}{t} = - \pdv{L}{t} \implies \dv{H}{t} \neq 0,
        \end{equation*}
        as the Lagrangian has an explicit time dependence. The pendulum is in an accelerating motion with the respect to the inertial frame of reference. This mean that $H$ will not be conserved.

    \section{Spherically symmetrical potential}
        Spherical coordinates defined by 
        \begin{equation*}
            x = r \sin(\theta) \cos(\varphi), \quad y = r \sin(\theta) \cos(\varphi), \quad z = r \cos(\theta)
        \end{equation*}
        This mean that the square velocity is 
        \begin{align*}
            &v^2 = \dot x^2 + \dot y^2 + \dot z^2 \\
            & = (\dot \sin(\theta) \cos(\vartheta) + r \dot \theta \cos(\theta) \cos(\vartheta) - r \dot \varphi \sin(\theta) \sin(\varphi))^2 \\
            & +  (\dot r \sin(\theta) \sin(\varphi) + r \dot \theta \cos(\theta) \sin(\varphi) + r \dot \varphi \sin(\theta)\cos(\varphi))^2 
            + (r \cos(\theta) - r \dot \theta \sin(\theta))^2 \\
            & = \underline{\dot r^2 \sin^2(\theta) \cos^2(\varphi)} + \dashuline{r^2 \dot \theta^2 \cos(\theta) \cos^2(\varphi)} + \dotuline{r^2 \dot \varphi^2 \sin^2(\theta) \sin^2(\varphi)} + \uuline{2 \dot r r \dot \theta \sin(\theta)\cos(\theta) \cos^2(\varphi)} \\
            & - \cancel{2 \dot r r \dot \varphi \sin^2(\theta) \cos(\varphi) \sin(\varphi)} - \cancel{2 r^2\dot \theta \dot \varphi \sin(\theta)\cos(\theta)\sin(\varphi)\cos(\varphi)} +  \underline{\dot r^2 \sin^2(\theta) \sin^2(\varphi)}  \\
            & + \dashuline{r^2 \dot \theta^2 \cos^2(\theta) \sin^2(\varphi)} + \dotuline{r^2 \dot \varphi^2 \sin^2(\theta) \cos^2(\varphi)} + \uuline{2 \dot r r \dot \theta \sin(\theta) \cos(\theta) \sin^2(\varphi)}\\
            &  + \cancel{2 \dot r r \dot \varphi \sin^2(\theta) \sin(\varphi) \cos(\varphi)}  + \cancel{2 r^2\dot \theta \dot \varphi \sin(\theta)\cos(\theta)\sin(\varphi)\cos(\varphi)}  + \dot r^2 \cos^2(\theta) + r^2 \dot \theta^2 \sin^2(\theta) \\
            & - 2 \dot r r \dot \theta \sin(\theta) \cos(\theta) \\
            & = \dot r^2 \sin^2(\theta) + r^2 \dot \theta^2 \cos^2(\theta) + r^2 \dot \varphi^2 \sin^2(\theta) + \cancel{2 \dot r r \dot \theta \sin(\theta)\cos(\theta)} \\
            &  + \dot r^2 \cos^2(\theta) + r^2 \dot \theta^2 \sin^2(\theta) - \cancel{2 \dot r r \dot \theta \sin(\theta) \cos(\theta) }\\
            & = \dot r^2 + (r \dot \theta)^2 + (r \dot \varphi \sin(\theta))^2
        \end{align*}
        The Lagrangian is 
        \begin{equation*}
            L = T - V = \frac{1}{2}m \left[\dot r^2 + (r \dot \theta)^2 + (r \dot \varphi \sin(\theta))^2\right] - \frac{k}{r},
        \end{equation*}
        so the canonical momenta are
        \begin{equation*}
            p_r = \pdv{L}{r} = m \dot r, \quad p_\theta = \pdv{L}{\theta} = m r^2 \dot \theta, \quad p_\vartheta = \pdv{L}{\varphi} = m r^2 \sin^2(\varphi) \dot \varphi.
        \end{equation*}
        This means we can rewrite the kinetic energy in terms of the momenta:
        \begin{equation*}
            T = \frac{1}{2m} \left[p_r^2 +\frac{p_\theta}{r^2} + \frac{p_\varphi}{r^2 \sin^2(\theta)}\right].
        \end{equation*}
        The Hamiltonian becomes
        \begin{equation*}
            H = T + V = \frac{1}{2m} \left[p_r^2 +\frac{p_\theta}{r^2} + \frac{p_\varphi}{r^2 \sin^2(\theta)}\right] - \frac{k}{r}.
        \end{equation*}
        Hamilton's equation of motion
        \begin{align*}
            & \dot r = \pdv{H}{p_r} = \frac{p_r}{m} \\
            & \dot \theta = \pdv{H}{p_\theta} = \frac{p_\theta}{m r^2} \\
            & \dot \varphi = \pdv{H}{p_\varphi} = \frac{p_\varphi}{m r^2 \sin^2(\theta)} \\
            & \dot p_r = - \pdv{H}{r} = \frac{p_\theta}{m r^3} + \frac{p_\varphi}{m r^3 \sin^2(\theta)} + \frac{k}{r^2} \\
            & \dot p_\theta = - \pdv{H}{\theta} = \frac{p_\varphi \cos(\theta)}{m r^2 \sin^3(\theta)} \\
            & \dot \varphi = - \pdv{H}{\varphi} = 0.
        \end{align*}


    \section{Earth's orbit}
        (Se compendium, chapter 4) \\ \\
        The eccentricity of a circle is $1$, so using the formula for the for the eccentricity of an orbit we get that
        \begin{equation*}
            \varepsilon = 0 = \sqrt{1 + \frac{2 E \ell^2}{m k^2}} \implies 1 + \frac{2 E \ell^2}{m k^2} = 0 \implies E = - \frac{m k^2}{2 \ell^2}.
        \end{equation*}
        On the other hand, for a circular orbit $E = V_{min}$. Using the effective 1D potential 
        \begin{equation*}
            E(r) = \frac{1}{2}m \dot r^2 + \frac{1}{2} \frac{\ell^2}{m r^2} - \frac{k}{r}.
        \end{equation*}
        If the mass of the sun is halved, then as the constant in the potential $k = GMm$ is proportional to the mass of the sun, it is halved, $k \rightarrow k/2$. This means the energy is change to
        \begin{equation*}
            E \rightarrow E' = \frac{1}{2}m \dot r^2 + \frac{1}{2} \frac{\ell^2}{m r^2} - \frac{k}{2r} = V_{min} + \frac{k}{2r} = - \frac{m k^2}{2 \ell^2} + \frac{k}{2 r}.
        \end{equation*}
        The original radius of the original orbit is given by the minimum of the effective potential,
        \begin{equation*}
            V(r) =  \frac{1}{2} \frac{\ell^2}{m r^2} - \frac{k}{r},
        \end{equation*}
        so
        \begin{equation*}
            V'(r) = -\frac{\ell^2}{m r^3} + \frac{k}{r^2} = 0 \implies r = \frac{\ell^2}{k m}.
        \end{equation*}
        This means the new energy is 
        \begin{equation*}
            E' = -\frac{m k^2 }{2 \ell^2} + \frac{mk^2}{2 \ell^2} = 0,
        \end{equation*}
        The new eccentricity is therefore 
        \begin{equation*}
            \varepsilon = \sqrt{1 + 0} = 1,
        \end{equation*}
        which means the new orbit is a parabola, and thus unbounded. The earth just escapes to infinity.


    \section{Einsteins correction}
        As the central force is given by
        \begin{equation*}
            - \frac{k}{r^2} + \frac{\beta}{r^3},
        \end{equation*}
        the potential is (up to a constant)
        \begin{equation*}
            V(r) = - \frac{k}{r} + \frac{\beta}{2 r^2}.
        \end{equation*}
        In the compendium, we can find that the angle of an object in a central potential is
        \begin{equation*}
            \theta(r) = \int_{r_0}^r \frac{1 / r^2 \, \dd r}{\sqrt{\frac{2 m E}{\ell^2} - \frac{2 m V(r)}{\ell^2} - \frac{1}{r^2}}}.
        \end{equation*}
        Inserting our potential, setting $u = 1/r$, and using $\gamma = 1 + \beta m / \ell^2$, this becomes
        \begin{equation*}
            \theta(r) = \int_{u_0}^u \frac{\dd u}{\sqrt{\frac{2 m E}{\ell^2} - \frac{2 m u}{\ell^2} - \gamma^2 u^2}}.
        \end{equation*}
        By introducing the constants
        \begin{equation*}
            a = \frac{2mE}{\ell^2}\quad b = \frac{2 m K}{\ell^2} \quad c^2 = -\gamma^2,
        \end{equation*}
        We get the integral on a known form which can be found in tables:
        \begin{equation*}
            \theta(r) = \int_{u_0}^u \frac{\dd u}{\sqrt{a + b u + c u^2}} = \frac{1}{\sqrt{-c}} \arccos \left(-\frac{b + 2c u}{\sqrt{b^2 - 4 a c}}\right).
        \end{equation*}
        Now,
        \begin{equation*}
            -\frac{b + 2c u}{\sqrt{b^2 - 4 a c}} = \frac{2\gamma^2u - 2mk/\ell^2}{\sqrt{\left(2 mk/\ell^2\right)^2 + 4\left(2mE/\ell^2\right)\gamma^2}} 
            = \frac{\frac{\ell^2 \gamma^2}{mk}u - 1}{\sqrt{1 + \frac{2E\gamma^2 \ell^2}{mk}}} = \frac{p / r - 1}{\sqrt{1 + \varepsilon}},
        \end{equation*}
        where
        \begin{equation*}
            p = \frac{\ell^2 \gamma^2}{mk}, \quad  \varepsilon = \frac{2E\gamma^2 \ell^2}{mk}.
        \end{equation*}
        This means the angle of the object is given by
        \begin{equation*}
            \theta(r) = \frac{1}{\gamma} \arccos\left(\frac{p/r - 1}{\sqrt{1 +\varepsilon}}\right).
        \end{equation*}
        Turning this around,
        \begin{equation*}
            \frac{p}{r} = 1 + \varepsilon\cos(\gamma \theta), \quad \mathrm{where}  \,\, \gamma = \sqrt{1 + \frac{m \beta}{\ell^2}} \approx 1 + \frac{m \beta}{2 \ell^2}, \, \, \frac{m \beta}{\ell} \ll 1.
        \end{equation*}
        If $E<0$, then this is an ellipse with slow precession. The semi-major axis for $\gamma = 1$ is
        \begin{equation*}
            a = \frac{p}{1 - \varepsilon^2} = \frac{\gamma^2 \ell^2 / m k}{1 - \left(1 + 2 E \gamma^2 \ell^2/ m k^2\right)} = \frac{k}{2 |E|}.
        \end{equation*}
        This, then, is a perturbation to this, with the smallness parameter $\eta = \beta/k a$, so $\gamma = 1 + m \eta k a / (2 \ell^2)$. For Mercury, $\eta = 1.42 \cdot 10^{-7}$, which is the perihelion precession of $43''$ per century.


\end{document}
