\documentclass{article}

\usepackage{physics} % Handy shortcuts like \pdv, \dd and much more
\usepackage{geometry} % smaller margins, can be adjusted if given arguments
\usepackage{siunitx} % the \si environment for units
\usepackage{mathtools} % The dcases environment, prettier than just cases
\usepackage{tikz} % For drawing picures
\usepackage{wrapfig} % Wrapping text around figures


\title{Exercise 7 - TFY4345 Classical Mechanics}
\date{2020}

\begin{document}
    \maketitle
    \section{Inertia tensor}
    A very thin rectangular slab has been placed in the $xyz$-coordinats system, such that the origin is in one of the slab corners, and the sides are along the $x$- and $y$-axes. The corresponding side lenghts are $a$ and $b$. Since the slab is very thin, we can assume that $z=0$ throughout.

    (NUMMERER a), b))
    \begin{itemize}
        \item Evaluate the individual elements of the inertia tensor.
        \item Set $a = b$, and solve for the principal moments of inertia, and the corresponding principal axes.
    \end{itemize}

\end{document}
