\documentclass{article}

\usepackage{physics} % Handy shortcuts like \pdv, \dd and much more
\usepackage{geometry} % smaller margins, can be adjusted if given arguments
\usepackage{siunitx} % the \si environment for units
\usepackage{mathtools} % The dcases environment, prettier than just cases
\usepackage{tikz} % For drawing picures
\usepackage{wrapfig} % Wrapping text around figures


\title{Exercise 5 solutions - TFY4345 Classical Mechanics}
\date{2020}

\begin{document}
    \maketitle
    \section{Effective potential and scattering center}
        The total energy, as given by equation 4.14 in the compendium, is
        \begin{equation*}
            E = \frac{1}{2}m\left( \dot r^2 + (r\dot \theta)^2 \right) + V(r).
        \end{equation*}
        In a central potential, we have that $mr^2 \dot \theta = \ell$  is a conserved quantity, so we get 
        \begin{equation*}
            E = \frac{1}{2} m \dot r^2 + \left(\frac{\ell^2}{2 m r^2} + V(r) \right) = \frac{1}{2}m \dot r^2 + V_{\mathrm{eff}}(r).
        \end{equation*}
        This is an effective 1D problem, with an effective potential 
        \begin{equation*}
            V_{\mathrm{eff}}(r) = V(r) + \frac{\ell^2}{2 m r^2}
        \end{equation*}
        In order for the particle to reach the center, it need to have sufficiently high energy to overcome the potential barrier, i.e. $E > V_{\mathrm{eff}}(r \rightarrow 0)$. This can be written as
        \begin{equation*}
            E r^2 > r^2 V(r) + \frac{\ell^2}{2 m}, \quad r \leftarrow 0.
        \end{equation*}
        The l.h.s. goes to zero, so that the condition becomes 
        \begin{equation*}
            (r^2V(r))_{r \rightarrow 0} < - \frac{\ell^2}{2m}.
        \end{equation*}
        This can be fulfilled wither with $- k / r^2$, where $k > \ell^2 / 2m$, or if $V(r) = - A/r^n$, with $n > 2$ and $A$ a positive constant.

    \section{Scattering from a spherical obstacle}
        (FIGUR) \\ \\
        The scattering angle $\theta$ satisfies $2 \Psi + \theta = \pi$. From the figure, we see that the impact parameter is given by $s = a \sin(\pi/2 - \theta /2 ) = a \cos(\theta / 2)$, so that 
        \begin{equation*}
            \left| \dv{s}{\theta} \right| = \frac{a}{2} \sin\left(\frac{\theta}{2}\right)
        \end{equation*}
        Using the formula for the differential cross section, as given in equation 4.40 in the compendium, we get
        \begin{equation*}
            \sigma(\theta) = \frac{s}{\sin(\theta)} \left| \frac{s}{\theta} \right| = \frac{a^2}{4}.
        \end{equation*}
        The total cross section is therefore
        \begin{equation*}
            \sigma = 2 \pi \int_0^{\pi} \sin(\theta) \sin(\theta) \dd \theta = \pi a^2.
        \end{equation*}
        This is physically sensible, since it is the actual cross-sectinoal area of the sphere.



    \section{Scattering by an attractive hard sphere}
        (FIGURE) \\ \\
        The larges impact parameter $s_max$ will send the particle just gracing the surface at $r=a$. Due to conservation of energy, we have that 
        \begin{equation*}
            E = \frac{1}{2} m v_0^2 = \frac{1}{2}mv^2 - \frac{k}{a}.
        \end{equation*}
        Furthermore, conservation of angular momentum means that $\ell$ infinitely far away is the same as when the particle touches the surface, so
        \begin{equation*}
            \ell = m v_0 s_\mathrm{max} = mva.
        \end{equation*}
        Combining thes two equations, we get
        \begin{equation*}
            s_\mathrm{max} = \frac{v}{v_0}a = a \sqrt{1 + \frac{2 k}{m a v_0^2}}.
        \end{equation*}
        All particles with impact parameter $s < s_\mathrm{max}$ will hit the surface, so that $\sigma_\mathrm{eff} = \pi s_\mathrm{max}^2$.


    \section{Average energies in the Kepler problem}
    

\end{document}

