\documentclass{article}

\usepackage{physics} % Handy shortcuts like \pdv, \dd and much more
\usepackage{geometry} % smaller margins, can be adjusted if given arguments
\usepackage{siunitx} % the \si environment for units
\usepackage{mathtools} % The dcases environment, prettier than just cases
\usepackage{tikz} % For drawing picures
\usepackage{wrapfig} % Wrapping text around figures
\usepackage{cancel} % Strikekethrough parts of equations


\title{Exercise 1 solutions - TFY4345 Classical Mechanics}
\date{2020}

\begin{document}
    \maketitle
    \section{Halley's comet}
        a) The gravitational force on the comet is
        \begin{equation*}
            \vec F = - \frac{GmM}{r^2} \vec e_r,
        \end{equation*}
        where $m$ is the mass of the comet, and $M$ is the mass of the sum. The torque on the comet is therefore
        \begin{equation*}
            \vec N = \vec r \times \vec F = - r \vec e_r \times \frac{GmM}{r^2} \vec e_r = \frac{GmM}{r} \vec e_r \times \vec e_r = 0. 
        \end{equation*}
         b) The angular momentum of the comet when the comet is closest to and farthest from the sun is, respectively $\vec L_e = \vec r_e \times \vec p_e$ and $\vec L_f = \vec r_f \times \vec p_f$. Conservation of angular momentum implies that $\vec L_e = \vec L_f$, and in turn that $\vec |L_e| = |\vec L_f|$. At both the point the comet is closest to and farthest from the sun, the position vector $\vec r$ and the momentum vector $\vec p$ are perpendicular. This means that
        \begin{equation*}
            \vec |L_e| = |\vec L_f| \implies |\vec r_e || \vec p_e| = |\vec r_f || \vec p_f| \implies r_e m v_e = r_f m v_f.
        \end{equation*}
        The velocity of the comet farthest from the sun is therefore
        \begin{equation*}
            v_f = \frac{r_e}{r_f} v_e = \frac{\SI{0.6}{AU}}{\SI{35}{AU}} \, \SI{54}{km.\per.s} = \SI{0.9}{km.\per.s} 
        \end{equation*}

    \section{Simple pendulum}
        a) By a trigonometric consideration,
        \begin{equation*}
            \vec R = \ell \sin(\beta) \vec e_x - \ell \cos(\beta) \vec e_y
        \end{equation*}

        b) Potential energy for a mass $m$ in a uniform gravitational field is $V = mgh$, where $h$ is the height of the mass (in an arbitrary reference frame). In our case, $\ell \cos(\beta)$, and hence
        \begin{equation*}
            V = - m g \ell \cos(\beta).
        \end{equation*}
        A different choice of reference frame will only add a constant to $V$, which will not affect the equations of motion. \\ \\
        c) The velocity of the mass $m$ is given by
        \begin{equation*}
            \vec v = \dv{\vec R}{t} = \ell \cos(\beta) \dot \beta \vec e_x + \ell \sin(\beta) \dot \beta \vec e_y.
        \end{equation*}
        The square of the velocity is then given by
        \begin{equation*}
            v^2 = \vec v \cdot \vec v = (\ell \dot \beta)^2 (\cos^2(\beta) \vec e_x \cdot \vec e_x + \sin^2(\beta) \vec e_y \cdot \vec e_y ) = (\ell \dot \beta)^2.
        \end{equation*}
        The kinetic energy is therefore 
        \begin{equation*}
            T = \frac{1}{2} m v^2 = \frac{1}{2}m (\ell \dot \beta)^2  
        \end{equation*}
        d) The Lagrangian of the pendulum is 
        \begin{equation*}
            L = T - V = \frac{1}{2}m (\ell \dot \beta)^2  + m g \ell \cos(\beta).
        \end{equation*}
        The Lagrange equations are in general
        \begin{equation*}
            \dv{t} \pdv{L}{\dot q_i} - \pdv{L}{q_i} = 0.
        \end{equation*}
        In this case, we have only one variable, $q_1 = \beta$. The resulting Lagrange equation is 
        \begin{equation*}
            \dv{t} \pdv{L}{\dot \beta} - \pdv{L}{\beta} = 0.
        \end{equation*}
        Inserting the Lagrangian we found, we get
        \begin{align*}
            \dv{t}\pdv{L}{\dot \beta} 
            = \dv{t} \left( m \ell^2 \dot \beta \right) 
            = m \ell^2 \ddot \beta, 
            \quad \pdv{L}{\beta} = - m g \ell \sin(\beta)    
        \end{align*}
        so the equation of motion is
        \begin{equation*}
            \ddot \beta + \frac{g}{\ell} \cos(\beta) = 0
        \end{equation*}

    \section{Double pendulum}
        (a) The position of mass $m_1$ is, just as in the single pendulum, 
        \begin{equation*}
            \vec R_1 = \ell_1 \sin(\beta_1) \vec e_x - \ell_1 \cos(\beta_1) \vec e_y.
        \end{equation*}
        The position of mass two is then, by the same considerations just using $\vec R_1$ as the point of reference,
        \begin{align*}
            \vec R_2 & = \vec R_1 + \ell_2 \sin(\beta_2) \vec e_x - \ell_2 \cos(\beta_2) \vec e_y \\
            &= \left[\ell_1 \sin(\beta_1) + \ell_2 \sin(\beta_2)\right] \vec e_x - \left[\ell_1 \cos(\beta_1) + \ell_2 \cos(\beta_2)\right] \vec e_y.
        \end{align*}
        The velocity and square velocity of $m_1$ is also just as in exercise 2:
        \begin{equation*}
            \vec v_1 = \dv{R_1}{t}, \quad v_1^2 = (\ell_1 \dot \beta_1)^2.
        \end{equation*}
        The velocity  and velocity square of $m_2$ is
        \begin{align*}
            \vec v_2 &= \dv{\vec R_2}{t} = 
            \left[\ell_1 \cos(\beta_1) \dot \beta_1 + \ell_2 \cos(\beta_2) \dot \beta_2 \right] \vec e_x + \left[\ell_1 \sin(\beta_1) \dot \beta_1 + \ell_2 \sin(\beta_2) \dot \beta_2\right] \vec e_y, \\
            v_2^2 &= \left[\ell_1 \cos(\beta_1) \dot \beta_1 + \ell_2 \cos(\beta_2) \dot \beta_2 \right]^2 + \left[\ell_1 \sin(\beta_1) \dot \beta_1 + \ell_2 \sin(\beta_2) \dot \beta_2\right]^2 \\ 
            &= (\ell_1 \dot \beta_1)^2 + (\ell_2 \dot \beta_2)^2 + 2 \ell_1 \ell_2 \cos(\beta_1)\cos(\beta_2) \dot \beta_1 \dot \beta_2 + 2 \ell_1 \ell_2 \sin(\beta_1)\sin(\beta_2) \dot \beta_1 \dot \beta_2 \\
            &= (\ell_1 \dot \beta_1)^2 + (\ell_2 \dot \beta_2)^2 + 2 \ell_1 \ell_2 \dot \beta_1 \dot \beta_2 \cos(\beta_1 - \beta_2),
        \end{align*}
        where we have used the trigonometric addition law $\cos(\theta) \cos(\phi) + \sin(\theta) \sin(\phi) = \cos(\theta - \phi)$ in the last line. Total potential energy is 
        \begin{equation*}
            V = m_1 g h_1 + m_2 g h_2 = -m_1 g \ell_1 \cos(\beta_1) - m_2 g \left[\ell_1 \cos(\beta_1) + \ell_2 \cos(\beta_2)\right].
        \end{equation*}
        Total kinetic energy is
        \begin{align*}            
            T &= \frac{1}{2}m_1 v_1^2 + \frac{1}{2}m_2 v_2^2 = \frac{1}{2}m_1 (\ell_1 \dot \beta_1)^2 + \frac{1}{2}m_2 \left[ (\ell_1 \dot \beta_1)^2+  (\ell \dot \beta_2)^2 + \ell_1 \ell_2 \dot \beta_1 \dot \beta_2 \cos(\beta_1 - \beta_2)\right],
        \end{align*}
        so the Lagrangian is, after gathering some terms, 
        \begin{align*}
            L &= T - V  \\
            &= (m_1 + m_2) \left[\frac{1}{2}(\ell_1 \dot \beta_1)^2 + g \ell_1 \cos(\beta_1)  \right] 
            + m_2 \left[\frac{1}{2} (\ell \dot \beta_2)^2 + \ell_1 \ell_2 \dot \beta_1 \dot \beta_2 \cos(\beta_1 - \beta_2) + g \ell_2 \cos(\beta_2)\right].
        \end{align*}
        b) The Lagrange equations for this problem are
        \begin{align*}
            & \dv{t} \pdv{L}{\dot \beta_1} - \pdv{L}{\beta_1} = 0, \\
            & \dv{t} \pdv{L}{\dot \beta_2} - \pdv{L}{\beta_2} = 0.
        \end{align*}
        Calculating the quantities needed:
        \begin{align*}
            & \pdv{L}{\beta_1} = -(m_1 + m_2) g \ell_1 \sin(\beta_1) - m_2 \ell_1 \ell_2 \dot \beta_1 \dot \beta_2 \sin(\beta_1 - \beta_2) \\
            & \pdv{L}{\dot \beta_1} = (m_1 + m_2) \ell_1^2 \dot \beta_1 + m_2 \ell_1 \ell_2 \dot \beta_2 \cos(\beta_1 - \beta_2) \\
            & \dv{t} \pdv{L}{\dot \beta_1} = (m_1 + m_2) \ell_1^2 \ddot \beta_1 + m_2 \ell_1 \ell_2 \left( \ddot \beta_2 \cos(\beta_1 - \beta_2) - \dot \beta_2 \sin(\beta_1 - \beta_2)(\dot \beta_1 - \dot \beta_2) \right)\\
            & \pdv{L}{\beta_2} =  m_2 \left[ \ell_1 \ell_2 \dot \beta_1 \dot \beta_2 \sin(\beta_1 - \beta_2)  - g \ell_2 \sin(\beta_2)\right]\\
            & \pdv{L}{\dot \beta_2} = m_2 \left[\ell_2^2 \dot \beta_2 + \ell_1 \ell_2 \dot \beta_1 \cos(\beta_1 - \beta_2)\right] \\
            & \dv{t} \pdv{L}{\dot \beta_2} = m_2 \left[ \ell_2^2 \ddot \beta_2 + \ell_1 \ell_2 \left(\ddot \beta_1\cos(\beta_1 - \beta_2) - \dot \beta_1 \sin(\beta_1 - \beta_2) (\dot \beta_1 - \dot \beta_2)\right) \right].
        \end{align*}
        This gives us the equation of motion    
        \begin{align*}
              &(m_1 + m_2) \ell_1^2 \ddot \beta_1 + m_2 \ell_1 \ell_2 \left( \ddot \beta_2 \cos(\beta_1 - \beta_2) - \dot \beta_2 \sin(\beta_1 - \beta_2)(\dot \beta_1 - \dot \beta_2) \right) \\
              &+ (m_1 + m_2) g \ell_1 \sin(\beta_1) + m_2 \ell_1 \ell_2 \dot \beta_1 \dot \beta_2 \sin(\beta_1 - \beta_2) = 0 \\
        \end{align*}
        and 
        \begin{align*}
            & m_2 \left[ \ell_2^2 \ddot \beta_2 + \ell_1 \ell_2 \left(\ddot \beta_1\cos(\beta_1 - \beta_2) - \dot \beta_1 \sin(\beta_1 - \beta_2) (\dot \beta_1 - \dot \beta_2)\right) \right]  \\
            & - m_2 \left[ \ell_1 \ell_2 \dot \beta_1 \dot \beta_2 \sin(\beta_1 - \beta_2)  - g \ell_2 \sin(\beta_2)\right] = 0 \\
        \end{align*}
        Cleaning up some, we get the final result
        \begin{align*}
            (m_1 + m_2) \left[\ell_1 \ddot \beta_1 + g \sin(\beta_1)\right]+ m_2 \ell_2 \left[ \ddot \beta_2 \cos(\beta_1 - \beta_2) + \dot \beta_2^2 \sin(\beta_1 - \beta_2) \right] = 0& \\
            \ell_2 \ddot \beta_2 + \ell_1 \ddot \beta_1 \cos(\beta_1 - \beta_2) - \ell_2 \dot \beta_1^2 \sin(\beta_1 - \beta_2) + g \sin(\beta_2) = 0&.
        \end{align*}

        \section{Lagrangian invariance}
        The original Lagrangian, $L(q, \dot q, t)$, gives the equation of motion
        \begin{equation*}
            \dv{t} \pdv{L}{\dot q} - \pdv{L}{q} = 0,
        \end{equation*}
        while the new Lagrangian, $L'(q, \dot q, t)$ gives
        \begin{align*}
            &\dv{t} \pdv{L}{\dot q} - \pdv{L}{q} = 
            \dv{t}\pdv{\dot q} \left( L(q, \dot q, t) + \dv{F(q, t)}{t} \right) - \pdv{q} \left( L(q, \dot q, t) + \dv{F(q, t)}{t} \right) = 0, \\
            &\implies \dv{t} \pdv{L}{\dot q} - \pdv{L}{q} + \dv{t} \pdv{\dot q} \dv{F(q, t)}{t}  - \pdv{q} \dv{F(q, t)}{t} = 0,
        \end{align*}
        This means the new Lagrangian will give us the same equation of motion, if 
        \begin{equation}
            \dv{t} \pdv{\dot q} \dv{F(q, t)}{t}  - \pdv{q} \dv{F(q, t)}{t} = 0.
            \label{eq1}
        \end{equation}
        We can use the chain rule to obtain
        \begin{equation*}
            \dv{F(q, t)}{t} = \pdv{F}{t} + \pdv{F}{q} \dot q.
        \end{equation*}
        This means that
        \begin{align*}
            \pdv{q} \dv{F}{t} = \pdv{F}{q}{t} + \pdv[2]{F}{q} \dot q, \\
            \dv{t} \pdv{\dot q} \dv{F}{t} = \dv{t} \left(\pdv{F}{q}\right) = \pdv{F}{t}{q} + \pdv[2]{F}{q}\dot q.
        \end{align*}
        Inserting this into \eqref{eq1}, we get the desired result:
        \begin{equation*}
            \pdv{F}{t}{q} + \pdv[2]{F}{q}\dot q - \left(\pdv{F}{q}{t} + \pdv[2]{F}{q} \dot q \right) = 0.
        \end{equation*} 
\end{document}

