\documentclass{article}

\usepackage{physics} % Handy shortcuts like \pdv, \dd and much more
\usepackage{geometry} % smaller margins, can be adjusted if given arguments
\usepackage{siunitx} % the \si environment for units
\usepackage{mathtools} % The dcases environment, prettier than just cases
\usepackage{tikz} % For drawing picures
\usepackage{wrapfig} % Wrapping text around figures
\usepackage{enumitem} % Getting alphabetical enumerate


\title{Exercise 9 solutions - TFY4345 Classical Mechanics}
\date{2020}

\begin{document}
    \maketitle
    \section{Coupled pendula}
    (FIGUR) \\
    The kinetic energy of the masses are 
    \begin{equation*}
        T = \frac{1}{2}m\left[(b \dot \theta_1)^2 + (b \dot \theta_2)^2\right].
    \end{equation*}
    The displacement of the masses in the vertical direction is given by $b(1 - \cos(\theta))$. As we are considering small oscillations, we only care about the stretching of the spring due to the horizontal movement of the pendula. This is given by $b (\sin(\theta_1) - \sin(\theta_2))$. Thus, the potential energy of the system is
    \begin{equation*}
        U = mgb\left[(1 - \cos(\theta_1)) + (1 - \cos(\theta_2))\right] + \frac{1}{2}kb^2\left[\sin(\theta_1) - \sin(\theta_2)\right]^2.
    \end{equation*}
    Using the small angle approximation, we get $\sin(\theta) \approx \theta$, $1 - \cos(\theta) \approx \frac{1}{2} \theta^2$, so the potential energy becomes
    \begin{equation*}
        \frac{1}{2} mgb (\theta_1^2 + \theta_2^2) - \frac{1}{2}kb^2(\theta_1 - \theta_2).
    \end{equation*}

    \section{Two coupled oscillators}
    (FIGUR) \\

    \section{Oscillating body with two attached pendula}
    (FIGUR) \\

    \section{Double pendulum}
    (FIGUR) \\

\end{document}
