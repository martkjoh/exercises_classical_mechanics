\documentclass{article}

\usepackage{physics} % Handy shortcuts like \pdv, \dd and much more
\usepackage{geometry} % smaller margins, can be adjusted if given arguments
\usepackage{siunitx} % the \si environment for units
\usepackage{mathtools} % The dcases environment, prettier than just cases
\usepackage{tikz} % For drawing picures
\usepackage{wrapfig} % Wrapping text around figures
\usepackage{enumitem} % Getting alphabetical enumerate


\title{Exercise 8 solutions - TFY4345 Classical Mechanics}
\date{2020}

\begin{document}
    \maketitle
    \section{Principal moments of inertia of a triangular slab}
    \begin{enumerate}[label=(\alph*)]
        \item Since the mas has uniform density, we can write the mass area density as $M =   1/2 ab\rho$. Let $x_{CM}$ denote the $x$-component of the center of mass. Using the definition of $CM$, we find (EXPLAIN UPPER LIMIT?)
        \begin{equation*}
            x_{CM} = \frac{1}{M} \int_0^a \dd x \int_0^{b(1 - x / a)} \dd y \rho x = \frac{\rho b}{M} \int_0^a \dd x \left( 1 - \frac{x}{a}\right) = \frac{a^2 b \rho}{M} \int_0^1 \dd u (1 - u)u = \frac{\rho a^2 b}{6M} = \frac{a}{3}. 
        \end{equation*}
        We used the substitution $u = 1 - x/a$ which implies a $ \dd x = - a \dd u $. Because of the symmetry in the problem (the slab is a triangle), the calculation of $y_{CM}$ is the same, only exchanging $ a \leftrightarrow b $, so the result is $y_{CM}$.
        \item The slab is two dimensional, and laying in the $xy$-plane. If we look at the definition of the off-diagonal entries in moment of inertia tensor,
        \begin{equation*}
            I_{ij} = - \int_V \dd V x_ix_j,
        \end{equation*} $I_{zx} = I_{xz} = I_{zy} = I_{yz} = 0$, as  z = 0. This also implies that $I_{xx} + I_{yy} = I_{zz}$, so all we need to calculate is $I_{xx}, I_{yy}$ and $I_{xy} ) I_{yx}$.
        \begin{align*}
            I_{xy} =& -\rho \int_0^a \dd x \int_0^{v(1 - x/a)} \dd y yx = - \frac{\rho b^2}{2}\int_0^a \dd x x \left( 1 - \frac{x}{a} \right)^2 = - \frac{\rho b^2}{2} \int_0^a \dd x \left( x - \frac{2}{a}x^2 + \frac{1}{a^2} x^3\right) \\
            &= -\frac{\rho b^2 a^2}{2}\left( \frac{1}{2} - \frac{2}{3} + \frac{1}{4} \right) = \frac{M a b}{12} \\
            I_{xy} = & -\rho \int_0^a \dd x \int_0^{v(1 - x/a)} \dd y y^2 = \frac{\rho b^3}{3}
            \left(1 - \frac{x}{a}\right)^3 = \frac{\rho a b^3}{3} \int_0^1 \dd u u^3 = \frac{M b^2}{6}.
        \end{align*} 
        Lastly, $I_{yy}$ can a gain be found just by the exchange $a \leftrightarrow b$. In matrix form,
        \begin{equation*}
            I = \frac{M}{6} = 
            \begin{pmatrix*}
                b^2 & -\frac{1}{2} ab & 0 \\
                -\frac{1}{2} ab & a^2 & 0 \\
                0 & 0 & a^2 + b^2 \\
            \end{pmatrix*}
        \end{equation*}
        \item We can remove the common factor $M/6$, so insert our values into the new variables, we get
        \begin{equation*}
            A = \frac{1}{2}(a^2 + b^2), \quad B = \frac{1}{2}\sqrt{(b^2 - a^2) +a^2b^2}, \quad \vartheta = \tan^{-1}\left( \frac{ab}{b^2 - a^2} \right).
        \end{equation*}
        (FIGUR) The last equation describes a traingle with side lengths $b^2 - a^2, \, ab$ and $\sqrt{(b^2 - a^2) +a^2b^2}$, and an angle $\vartheta$ opposite the side of length $ab.$ 

    \end{enumerate}
    \section{Precession of a frisbee}
    \section{Precession of a heavy spinning top}
\end{document}
