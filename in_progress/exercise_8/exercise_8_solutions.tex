\documentclass{article}

\usepackage{physics} % Handy shortcuts like \pdv, \dd and much more
\usepackage{geometry} % smaller margins, can be adjusted if given arguments
\usepackage{siunitx} % the \si environment for units
\usepackage{mathtools} % The dcases environment, prettier than just cases
\usepackage{tikz} % For drawing picures
\usepackage{wrapfig} % Wrapping text around figures
\usepackage{enumitem} % Getting alphabetical enumerate


\title{Exercise 8 solutions - TFY4345 Classical Mechanics}
\date{2020}

\begin{document}
    \maketitle
    \section{Principal moments of inertia of a triangular slab}
    \begin{enumerate}[label=(\alph*)]
        \item Since the mas has uniform density, we can write the mass area density as $M =   1/2 ab\rho$. Let $x_{CM}$ denote the $x$-component of the center of mass. Using the definition of $CM$, we find (EXPLAIN UPPER LIMIT?)
        \begin{equation*}
            x_{CM} = \frac{1}{M} \int_0^a \dd x \int_0^{b(1 - x / a)} \dd y \rho x = \frac{\rho b}{M} \int_0^a \dd x \left( 1 - \frac{x}{a}\right) = \frac{a^2 b \rho}{M} \int_0^1 \dd u (1 - u)u = \frac{\rho a^2 b}{6M} = \frac{a}{3}. 
        \end{equation*}
        We used the substitution $u = 1 - x/a$ which implies a $ \dd x = - a \dd u $. Because of the symmetry in the problem (the slab is a triangle), the calculation of $y_{CM}$ is the same, only exchanging $ a \leftrightarrow b $, so the result is $y_{CM}$.
        \item The slab is two dimensional, and laying in the $xy$-plane. If we look at the definition of the off-diagonal entries in moment of inertia tensor,
        \begin{equation*}
            I_{ij} = - \int_V \dd V x_ix_j,
        \end{equation*} $I_{zx} = I_{xz} = I_{zy} = I_{yz} = 0$, as  z = 0. This also implies that $I_{xx} + I_{yy} = I_{zz}$, so all we need to calculate is $I_{xx}, I_{yy}$ and $I_{xy}$.
        \begin{align*}
            I_{xy} =& -\rho \int_0^a \dd x \int_0^{v(1 - x/a)} \dd y yx = - \frac{\rho b^2}{2}\int_0^a \dd x x \left( 1 - \frac{x}{a} \right)^2 = - \frac{\rho b^2}{2} \int_0^a \dd x \left( x - \frac{2}{a}x^2 + \frac{1}{a^2} x^3\right) \\
            &= -\frac{\rho b^2 a^2}{2}\left( \frac{1}{2} - \frac{2}{3} + \frac{1}{4} \right) = \frac{M a b}{12} \\
            I_{xy} = & -\rho \int_0^a \dd x \int_0^{v(1 - x/a)} \dd y y^2 = \frac{\rho b^3}{3}
            \left(1 - \frac{x}{a}\right)^3 = \frac{\rho a b^3}{3} \int_0^1 \dd u u^3 = \frac{M b^2}{6}.
        \end{align*} 
        Lastly, $I_{yy}$ can a gain be found just by the exchange $a \leftrightarrow b$. In matrix form,
        \begin{equation*}
            I = \frac{M}{6}
            \begin{pmatrix*}
                b^2 & -\frac{1}{2} ab & 0 \\
                -\frac{1}{2} ab & a^2 & 0 \\
                0 & 0 & a^2 + b^2 \\
            \end{pmatrix*}
        \end{equation*}
        \item We can remove the common factor $M/6$, so insert our values into the new variables, we get
        \begin{equation*}
            A = \frac{1}{2}(a^2 + b^2), \quad B = \frac{1}{2}\sqrt{(b^2 - a^2) +a^2b^2}, \quad \vartheta = \tan^{-1}\left( \frac{ab}{b^2 - a^2} \right).
        \end{equation*}
        (FIGUR)\\
        The last equation describes a triangle with side lengths $b^2 - a^2, \, ab$ and $\sqrt{(b^2 - a^2) +a^2b^2} = 2B$, and an angle $\vartheta$ opposite the side of length $ab$. This gives us the relations $ab=2 B \cos(\vartheta)$ and $b^2 - a^2 = 2B \cos(\vartheta)$. (HER ER DET NOE FEIL) It follows that
        \begin{align*}
            a^2 &= \frac{1}{2}(b^2 + a^2)  - \frac{1}{2}(b^2 - a^2) = A - B \cos(\vartheta) \\
        b^2 &= \frac{1}{2}(b^2 + a^2)  + \frac{1}{2}(b^2 - a^2) = A + B \cos(\vartheta)
        \end{align*}
        Putting all this together, we get 
        \begin{equation*}
            I = \frac{M}{18}
            \begin{pmatrix*}
                A + B\cos(\vartheta) & B \sin(\vartheta) & 0 \\
                B \sin(\vartheta) & A - B\cos(\vartheta) & 0 \\
                0 & 0 & 2A
            \end{pmatrix*}
        \end{equation*}

        To find the principal moments of inertia, we must find solve the characteristic equation for the principal moments of inertia $\omega$
        \begin{align*}
             \det(I - \omega) = 0& \implies
            \begin{vmatrix*}
                A + B\cos(\vartheta) - \omega & B \sin(\vartheta) & 0 \\
                B \sin(\vartheta) & A - B\cos(\vartheta) - \omega & 0 \\
                0 & 0 & 2A - \omega
            \end{vmatrix*} \\
            &= (2A - \omega) \big[(A + B\cos(\vartheta) - \omega) (A - B\cos(\vartheta) - \omega)  - B \sin(\vartheta) \big]  \\
            &= (2A - \omega)[A^2 - B^2 + \omega^2 - 2\omega A] \\
            &= (2A - \omega)[(A - \omega)^2 - B^2] = 0,
        \end{align*}
        which has the solutions $\omega_1 = 2A, \, \omega_2 = A+B$ and $\omega_3 = A-B$. By inspection, the first eigenvector is $\mathbf{v} = (0, 0, 1)$. We can then only look at the relevant part of the matrix to find the others
        \begin{align*}
            \omega = A + B \implies& 
            \begin{pmatrix*}
                2A + B(1 + \cos(\vartheta)) & B \sin(\vartheta)  \\
                B \sin(\vartheta) & 2A + B(1-\cos(\vartheta))  \\
            \end{pmatrix*} 
            \mathbf{v}\\
            =& 
            \begin{pmatrix*}
                2[A + B \cos^2(\vartheta / 2)] & B \sin(\vartheta)  \\
                B \sin(\vartheta) & 2[A + B\sin^2(\vartheta/2)]  \\
            \end{pmatrix*} 
            \mathbf{v} = 0 \\
            &\implies 0 = 
            \begin{cases}
                2[A + B \cos^2(\vartheta / 2)] v_1 + B \sin(\vartheta) v_2 \\
                B \sin(\vartheta) v_1   + 2[A + B\sin^2(\vartheta/2)] v_2
            \end{cases}
        \end{align*}

    (TBD)
    \begin{equation*}
        \mathbf{v}_1 = (\cos(\vartheta/2), \sin(\vartheta/2), 0), \quad \mathbf{v}_1 = (-\sin(\vartheta/2), \cos(\vartheta/2), 0)
    \end{equation*}

    \end{enumerate}
    \section{Precession of a frisbee}

    \begin{enumerate}[label=(\alph*)]
    \item The Euler equation for the motion of a spinning free body (no torque) is 
        \begin{equation*}
            \left(\dv{\mathbf{L}}[t]\right)_b + \boldsymbol \omega \cross \mathbf{L} = 0
        \end{equation*}
        Writing this out in component form gives
        \begin{align*}
            & I_1 \dot \omega_{x'} + \omega_{y'}\omega_{z'}(I_3 - I_2) = 0, \\
            & I_2 \dot \omega_{y'} + \omega_{z'}\omega_{x'}(I_1 - I_3) = 0, \\
            & I_3 \dot \omega_{z'} + \omega_{x'}\omega_{y'}(I_2 - I_1) = 0 .\\
        \end{align*}
        As shown in the compendium (5.G), the components of the angular velocity in the body frame is 
        \begin{align*}
            \omega_{x'} & = \dot \phi \sin(\theta) \sin(\psi) + \dot\theta \cos(\psi) \\
            \omega_{y'} & = \dot \phi \sin(\theta)\cos(\psi) - \dot \theta  \sin(\psi)\\
            \omega_{z'} & = \dot \phi \cos(\theta) + \dot \psi.
        \end{align*}

    \item From the component form of the equations of motion, we see that 
        \begin{equation*}
            I_1 = I_2 \implies I_3 \dot \omega_{z'} = 0 \implies \omega_{z'} = \mathrm{const.}
        \end{equation*}

    \end{enumerate}

    \section{Precession of a heavy spinning top}
\end{document}
