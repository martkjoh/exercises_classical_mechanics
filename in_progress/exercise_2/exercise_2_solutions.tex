\documentclass{article}

\usepackage{physics} % Handy shortcuts like \pdv, \dd and much more
\usepackage{geometry} % smaller margins, can be adjusted if given arguments
\usepackage{siunitx} % the \si environment for units
\usepackage{mathtools} % The dcases environment, prettier than just cases
\usepackage{tikz} % For drawing picures
\usepackage{wrapfig} % Wrapping text around figures


\title{Exercise 2 solutions - TFY4345 Classical Mechanics}
\date{2020}

\newcommand{\Eff}{\mathcal{F}}

\begin{document}
    \maketitle
    \section{Damped oscillator}
        (a) The frictional force is 
        \begin{equation*}
            F_f = -\pdv{\mathcal{F}}{v}.
        \end{equation*}
        The work done by friction is force times distance, so the work per unit time is
        \begin{equation*}
            \dot W_f = - F_f v = \pdv{\mathcal{F}}{v} v \implies \mathcal{F} = C v^2.
        \end{equation*}
        (As $\mathcal{F}$ is a (velocity) potential, we can dismiss any constants, just as with regular potentials.) This means
        \begin{equation*}
            \dot W_f = 2 C v^2 = 2 \mathcal{F}. 
        \end{equation*}
        (b) The Lagrangian with a velocity-dependent potential is 
        \begin{equation*}
            \dv{t} \pdv{L}{\dot x} - \pdv{L}{x} + \pdv{\Eff}{\dot x} = 0.
        \end{equation*}
        Inserting the Lagrangian for a harmonic oscillator, 
        \begin{equation*}
            L = \frac{1}{2} m \dot x^2 - \frac{1}{2} k x^2,
        \end{equation*}
        and the given velocity potential $\Eff = 3 \pi \mu a \dot x^2$, we get
        \begin{align*}
            \dv{t} \pdv{L}{\dot x} = m \ddot x, \, \pdv{L}{x} = -k x, \, \pdv{\Eff}{\dot x} = 6 \pi \mu a \dot x, \\
            \implies m \ddot x + 6 \pi \mu a + k x = 0,
        \end{align*}
        or
        \begin{equation*}
            \ddot x + 2 \lambda \dot x + \omega^2_0x = 0, \quad \lambda = \frac{3 \pi \mu a}{m}, \, \omega_0 = \sqrt{\frac{k}{m}}.
        \end{equation*}
        (c) 

    \section{Operator identities}

    \section{Shortest path in polar coordinates}

    \section{Forces of constraint}


\end{document}

