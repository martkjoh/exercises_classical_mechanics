\documentclass{article}

\usepackage{physics} % Handy shortcuts like \pdv, \dd and much more
\usepackage{geometry} % smaller margins, can be adjusted if given arguments
\usepackage{siunitx} % the \si environment for units
\usepackage{mathtools} % The dcases environment, prettier than just cases
\usepackage{tikz} % For drawing picures
\usepackage{wrapfig} % Wrapping text around figures


\title{Exercise 2 solutions - TFY4345 Classical Mechanics}
\date{2020}

\newcommand{\Eff}{\mathcal{F}}

\begin{document}
    \maketitle
    \section{Damped oscillator}
        (a) The frictional force is 
        \begin{equation*}
            F_f = -\pdv{\mathcal{F}}{\dot x}.
        \end{equation*}
        The work done by friction is force times distance, so the work per unit time is
        \begin{equation*}
            \dot W_f = - F_f v = \pdv{\mathcal{F}}{\dot x} \dot x \implies \mathcal{F} = C \dot x ^2.
        \end{equation*}
        (As $\mathcal{F}$ is a (velocity) potential, we can dismiss any constants, just as with regular potentials.) This means
        \begin{equation*}
            \dot W_f = 2 C \dot x^2 = 2 \mathcal{F}. 
        \end{equation*}
        (b) The Lagrangian with a velocity-dependent potential is 
        \begin{equation*}
            \dv{t} \pdv{L}{\dot x} - \pdv{L}{x} + \pdv{\Eff}{\dot x} = 0.
        \end{equation*}
        Inserting the Lagrangian for a harmonic oscillator, 
        \begin{equation*}
            L = \frac{1}{2} m \dot x^2 - \frac{1}{2} k x^2,
        \end{equation*}
        and the given velocity potential $\Eff = 3 \pi \mu a \dot x^2$, we get
        \begin{align*}
            \dv{t} \pdv{L}{\dot x} = m \ddot x, \, \pdv{L}{x} = -k x, \, \pdv{\Eff}{\dot x} = 6 \pi \mu a \dot x, \\
            \implies m \ddot x + 6 \pi \mu a + k x = 0,
        \end{align*}
        or
        \begin{equation*}
            \ddot x + 2 \lambda \dot x + \omega^2_0x = 0, \quad \lambda = \frac{3 \pi \mu a}{m}, \, \omega_0 = \sqrt{\frac{k}{m}}.
        \end{equation*}
        (c) 
        If we assume the solution to be of the from
        \begin{equation*}
            x(t) = A e^{\omega_a t} + B e^{\omega_b t},
        \end{equation*}
        we get
        \begin{equation*}
            A(\omega_0^2 + 2 \lambda \omega_a + \omega^2)e^{\omega_a t} + B(\omega_0^2 + 2 \lambda \omega_b + \omega_b^2)e^{\omega_b t} = 0,
        \end{equation*}
        so
        \begin{equation*}
            \omega_{a/b} = -\lambda \pm \sqrt{\lambda^2 - \omega_0^2}
        \end{equation*}
        this gives us
        \begin{equation*}
            x(t) = e^{-\lambda t} \left( A \exp \left[\omega t \sqrt{(\lambda / \omega)^2 - 1} \right] + B \exp \left[- \omega t \sqrt{(\lambda / \omega)^2 - 1} \right] \right).
        \end{equation*}
        Now, as $\lambda/\omega \ll 1$, we get that $\sqrt{(\lambda / \omega) - 1} \approx i$. This means that, after applying the initial conditions, we get the solution
        \begin{equation*}
            x(t) = x_0 e^{-\lambda t} \cos(\omega t).
        \end{equation*}
        The instantaneous energy dissipation is therefore 
        \begin{equation*}
            \dot W_f = F_f \dot x = 2 \lambda m \dot x^2 = 2 \lambda m x_0^2 e^{-\lambda t} \cos^2(\omega t).
        \end{equation*}
        If we then take the average over a period $2 \pi / \omega_0$, we can assume the exponential is more or less constant (as $\lambda << \omega_0$), so the time averaged dissipation is 
        \begin{equation*}
            \overline{\dot W_f} = m \lambda (\omega_0 \dot x)^2 e^{-\lambda t}.
        \end{equation*}

    \section{Operator identities}
        The Einstein summation convention is used throughout this exercise, so repeated indeces are summed over. The $i$'th component of the curl of the curl of $\mathbf{A}$ can be written
        \begin{equation*}
            
        \end{equation*}

    \section{Shortest path in polar coordinates}

    \section{Forces of constraint}


\end{document}

