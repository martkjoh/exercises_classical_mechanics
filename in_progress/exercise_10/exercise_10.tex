\documentclass{article}

\usepackage{physics} % Handy shortcuts like \pdv, \dd and much more
\usepackage{geometry} % smaller margins, can be adjusted if given arguments
\usepackage{siunitx} % the \si environment for units
\usepackage{mathtools} % The dcases environment, prettier than just cases
\usepackage{tikz} % For drawing picures
\usepackage{wrapfig} % Wrapping text around figures


\title{Exercise 10 - TFY4345 Classical Mechanics}
\date{2020}

\begin{document}
    \maketitle
    \section{Velocity addition and Lorentz transformation matrices}
    Suppose three inertial systems $S, \, S'$ and $S''$ are moving with collinear motion along their respective $x_1$-axes. Let the velocity of $S'$ relative $S$ be $v_1$, and the velocity of $S''$ relative $S''$ be $v_2$. Write down the Lorentz transformation matrices $L$ and $L'$ corresponding to the transformations $S \rightarrow S'$ and $S' \rightarrow S''$. Use these to derive Einstein's addition rule based on the matrix elements of the transformation matrix $L''$ corresponding to $S \rightarrow S''$. \newline \newline % Jeg aner ikke hvorfor, men \\ fungerer ikke her...
    [See also Exam 2018 (December), problem 3, where $S'$ moves in $z$-direction and $S''$ moves in the $x'$ direction.] 
    \section{Light from a fluorescent tube}
        [Exam 2016] \\ \\
        A fluorescent tube lamp is stationary in a reference frame $S$, parallel to the $z$ axis. The tube lights up simultaneously (in $S$) along its entire length $L_0$ at the time $t=0$. (ER DET t=0 ELLER t?) The tube has one end at $z=0$, and the other at $z = L_0$. Consider an observer in a reference system $S'$ moving with a velocity $v$ in the $z$-axis. \\ \\
        a) We now consider two spacetime events in $S$, the lighting up o the to in position $z$ at time $t$, and in position $z + \Delta z$ also at time $t$. Use the lorentz transformation to calculate the spacetime coordinates of these two events in the $S'$ frame, ($z'$ at time $t'$) and ($z' + \Delta z'$ at time $t' + \Delta t'$). \\ \\
        b) For
\end{document}
